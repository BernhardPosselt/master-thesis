\chapter{Prozesserstellung}

sagen dass man eine art kochrezept will

\section{Vorhandene Daten}
Ausgangsparameter test
\subsection{Wer (Akteure}
\subsection{Was (Daten)}
\subsection{Warum (Anforderungen/Usecases)}
\subsection{Womit (Komponenten)}

\section{Prozesserstellungsversuche}
Ausgehend von den vorhandenen Daten wurden mehrere Prozesse definiert, welche bis auf den Letzten entweder zu grobe Ergebnisse lieferten, oder nicht nachvollziehbar waren.

Ausgangsbasis war eine Systemvision mit folgenden Anforderungen:

\begin{itemize}
  \item Es soll eine Webseite entstehen, welche die Prüfungstermine auflistet und Personen erlaubt, sich für diese Prüfungen anzumelden
  \item Die Übermittlung der Prüfungsdaten soll über einen eigenen VPN Server geschehen
  \item Die Prüfungsdaten werden firmenintern verwaltet und nach der Auswertung soll der Scheme Owner benachrichtigt werden
\end{itemize}

\begin{figure}[!htbp]
    \centering
    \includegraphics[scale=0.6]{uml/vision.png}
    \caption{Systemvision der Komponenten}
\end{figure}

Aufbauend darauf wurde dann versucht einen Prozess zu finden, der diese Grundideen berücksichtigt.

\subsection{Vom Usecase zur Komponente}
Der erste Versuch zur Erstellung des Architekturprozesses orientierte sich am Prinzip: teile und herrsche. Der Prozess hing stark von den nicht funktionalen Anforderungen ab und sollte auf folgender Weise funktionieren:

\begin{itemize}
  \item Für jeden Usecase wird ein komplettes Komponentendiagramm des Systems erstellt
  \item Die Komponenten jedes Teilsystems werden anhand ihrer nicht funktionalen Qualitäten aus einem Pool von Komponentenarchitekturen gewählt
  \item Schlussendlich werden alle Teilsysteme mit einander vereinigt, soweit es die nicht funktionalen Attribute erlauben
\end{itemize}

Dieser Prozess scheiterte nicht nur am enormen Modellierungsaufwand, sondern auch am Auswahlprozess der Komponenten: Je nachdem, welche Komponentenarchitekturen vorhanden waren und wie diese bewertet wurden, entstanden unterschiedliche Architekturen. Zudem schien es zu viele Komponentenarchitekturen zu geben, da die einzelnen Komponenten beliebig miteinander kombinierbar waren.

Die Qualität der Architektur hätte folglich von der Vollständigkeit dieser scheinbar unendlich großen Menge an Komponentenarchitekturen abgehangen. Aus diesem Grund schien er ungeeignet und wurde verworfen.

\subsection{Von einer erweiterten Systemvision auf eine verfeinerte Architektur durch nicht funktionale Anforderungen}
Um das Problem des sehr hohen Modellierungsaufwandes des ersten Prozesses zu umgehen, wurde von einer erweiterten Systemvision in Form eines Komponentendiagrammes ausgegangen. Diese Komponentenarchitektur entstand zusammen mit dem/der AuftraggeberIn.

\begin{figure}[!htbp]
    \centering
    \includegraphics[scale=0.5]{uml/vision2.png}
    \caption{Erweiterte, geschätzte Systemvision der Komponenten}
\end{figure}

Diese Architektur sollte nun anhand der nicht funktionalen Anforderungen jedes Usecases angepasst werden.

Auch dieser Prozess litt jedoch unter dem Problem, dass die nicht funktionalen Anforderungen schwer bewertet werden konnten. Über dies hinaus war es schwer ein Regelwerk/Rezept aus der Architekturerstellung abzuleiten, da durch die Einbeziehung unterschiedlicher AuftraggeberInnen jeweils verschiedene Architekturen entstehen können.

\subsection{Von den Daten zur Architektur}
Die Auswahl und Bewertung der Komponentenarchitekturen in den vorherigen beiden Prozessen stellte ein wesentliches Hinderniss zur Erstellung eines eindeutigen Regelwerkes dar. Eine vollständige Auflistung aller möglichen Komponentenarchitekturen erschien entweder als unmöglich oder unvollständig. Auch eine Bewertung schien ohne entsprechende Implementation unüberprüfbar zu sein.

Da sich dieser Prozess jedoch ausschließlich auf die Planungs- und nicht die Implementationsphase bezieht, wurden die nicht funktionalen Anforderungen und Komponentenarchitekturen als Hauptkriterium für die Architekturerstellung verworfen.

Stattdessen wurde der Fokus auf die Aufspaltung der Daten gelegt, welche bereits schon in den Systemvisionen der vorherigen Prozesse erkennbar war.

Die Daten wurden anhand Ihrer Vertraulichkeit in unterschiedliche Komponenten aufgeteilt. Diese Komponent wurden dann durch Systeme mit einander verbunden, die den Zugriff auf die Daten regelten.

\begin{figure}[!htbp]
    \centering
    \includegraphics[scale=0.7]{uml/vision3.png}
    \caption{Aufteilung der Komponenten in Datenbereiche}
\end{figure}

Dieser Prozess erlaubte es, eine nachvollziehbare Architektur zu erstellen, jedoch war das Ergebnis zu grob. Außerdem schien eine separate Komponente zur  Übertragung der Prüfungsdaten zu fehlen, welche in der ursprünglichen Systemvision definiert und als wichtig empfunden worden war.

\subsection{Von den Daten und den Akteuren zur Architektur}
Vierter: Einbeziehen der Akteure
