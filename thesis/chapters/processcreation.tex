\chapter{Prozesserstellung}
Beschreibung was für Ausgangsparameter vorhanden waren und welche Anläufe versucht wurden und warum sie fehl schlugen

\section{Vorhandene Daten}
Ausgangsparameter test
\subsection{Wer (Akteure}
\subsection{Was (Daten)}
\subsection{Warum (Anforderungen/Usecases)}
\subsection{Womit (Komponenten)}

\section{Prozesserstellungsversuche}
Ausgehend von den vorhandenen Daten wurden mehrere Prozesse definiert, welche bis auf den Letzten entweder zu grobe Ergebnisse lieferten, oder nicht nachvollziehbar waren.

Ausgangsbasis war eine Systemvision mit folgenden Anforderungen:

\begin{itemize}
  \item Es soll eine Webseite entstehen, welche die Prüfungstermine listet und Personen erlaubt, sich für diese Prüfungen anzumelden
  \item Die Übermittlung der Daten soll über einen eigenen VPN Server geschehen
  \item Die werden firmenintern verwaltet und nach der Auswertung soll der Scheme Owner benachrichtigt werden
\end{itemize}

\subsection{Vom Usecase zur Komponente}
Der erste Versuch zur Erstellung des Architekturprozesses orientierte sich am Prinzip: teile und herrsche. Zuerst

Erster Versuch: für jeden Usecase ein Teilsystem, vereinen anhand nicht funktionaler Anforderungen -> zu aufwendig, unendlich Komponenten
\subsection{Von einer geschätzten Architektur auf eine verfeinerte Architektur durch nicht funktionale Anforderungen}
Zweiter Versuch: Nach Erfahrungswerten so viel wie möglich erstellen, danach überprüfen auf nicht funktionale Anforderungen und notwenige Änderungen einbringen -> kein Kochrezept, verlässt sich zu viel auf Erfahrungswerte
\subsection{Von den Daten zur Architektur}
Dritter Versuch: Aufspalten der Architektur in Datenbereiche. Ging schon gut aber Ergebnis zu grob
\subsection{Von den Daten und den Akteuren zur Architektur}
Vierter: Einbeziehen der Akteure
