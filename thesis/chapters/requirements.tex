\chapter{Anforderungen}
Die Anforderungen des/der KundIn sind die Basis des zu entwickelnden Systems und beschreiben nicht nur dessen Funktionalität, sondern auch dessen Qualitäten. Sie werden vom Kunden meist in der Form von Usecases beschrieben \cite[S. 78]{reqanalysis} und sind für die Erstellung der Architektur notwendig \cite[S. 9]{softarch}. Ein besonderer Fokus bei der Erstellung der Architektur liegt auf der geforderten Qualität des Systems, da eben diese stark von Architekturentscheidungen beinflusst wird \cite[S. 109]{softarch}: \glqq Architektur ist für die Qualität eines Systems notwendig\grqq \cite[S. 59]{effektiv}.

Sowohl funktionale als auch qualitative Aspekte sind für den Erfolg des Systems essentiell \cite[S. 109]{softarch}. Dies lässt an folgendem Beispiel demonstrieren: Ein/Eine KundIn beauftragt eine Firma, einen Webshop zu erstellen, auf welchem ihre/seine Produkte verkauft werden sollen. Die Funktionalität des Webshops wird wie beschrieben implementiert, aber das Endprodukt ist so langsam, dass ein Großteil der KundInnen den Bestellvorgang abbricht. Der Hauptgrund der Erstellung des Systems, nämlich Produkte verkaufen, ist somit zwar möglich, aber für den Kunden unrentabel.


Da Was ist Qualität \cite[S. 399]{pract}

ISO 9126 \cite{ISO_SQ} und dessen Nachfolger ISO 25010 \cite{ISO_SQ2} bieten ein Modell, um Softwarequalität zu beschreiben. ISO 25010 erweitert die in ISO 9126 beschriebenen Hauptkategorien um Security und Compatibility und fügt eine Zusätzliche Kategorie, nämlich Software Quality in Use hinzu, welche die eine eigene Qualitätskategorie für BenutzerInnen definieren. Da es aber schwierig war, zitierbare Quellen zum neuen Standard zu finden - ISO 25010 hat \glqq in die Praxis wenig Einzug gehalten\grqq \ \cite[S. 60]{effektiv} - , die Neuerungen überschaubar und mehr eine Umorganisierung als Revolution darstellen, wird auf die Nutzung von ISO 25010 verzichtet und das Qualitätsmodell des Vorgängers, ISO 9216, verwendet.



Behandeln der ISO 9126 Anforderungen, versuchen zu erklären was man alles beachten muss und was überhaupt ab anfang messbar ist (z.B. performance nicht messbar)
Ein wichtiges Ziel der Softwarearchitektur ist die Sicherstellung der Qualität des Systems \cite[S. 19]{effektiv}:

*

\section{Funktionale Anforderungen}
beschreibt die funktion die notwendig ist um das system für den kunden wertvoll zu machen\cite[S. 79]{reqanalysis}


Wichtig: Security ist eine funktionale Anforderung, ISO 25000 + IREB

\section{Nicht funktionale Anforderungen}
Einschränkungen, beschreiben nicht was sondern wie\cite[S. 80]{reqanalysis}

\subsection{Reliability}
\subsection{Usability}
\subsection{Efficiency}
\subsection{Maintainability}
\subsection{Portability}