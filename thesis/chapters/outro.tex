\chapter{Zusammenfassung}

\section{Warum ist der Prozess gut}
\subsection{10er Regel der Fehlerkosten}
\subsection{Bezieht Architekturrelevante Anforderungen in den Anforderungsprozess ein}
\subsection{Minimiert Angriffsfläche für wichtige Infrastruktur}
\subsection{Fokus Datensicherheit}
\subsection{Lässt Entscheidungen aufschieben}
\subsection{Kochrezept lässt sich ableiten}
\subsection{Visuelle Repräsentationen mit denen schnell bewertet werden kann}

\section{Limitierungen}
\subsection{Benötigt noch mehr Input von anderen Projekten}
da beispielprojekt wegen den anforderungen starken wert auf datensicherheit legt und deswegen architektur prozess beinflusst hat, benötigt auch noch implmenentierungsphase und realtest in projekten

\subsection{Nicht alle nicht funktionalen Anforderungen überprüfbar}
Sprich es ist bis zu einem gewissen Teil möglich
\subsection{Problem wenn sicht Akteure/Daten oft ändern}
\subsection{Zum Teil auch abhängig von Erfahrungswerten}
\subsection{Nur Plangungsphase, ohne Implementierungsphase}
\subsection{Extremarchitekturen}

\section{Erkenntnisse}
\subsection{Ohne messbare Parameter kein Kochrezept möglich}
\subsection{Parameter nicht zu jeder Phase messbar}
\subsection{Priorisierung von nicht funktionalen Parametern schwer möglich}
\subsection{Generische Komponentenarchitektur durch viele und schnell ändernde Kombinationen schwer möglich}
\subsection{Auf Ehrfahrungswerte kann nicht vollkommen verzichtet werden}
\subsection{Funktionale Anforderungen beeinflussen Archtitektur mehr als gedacht}
Prozess erstellt die frühe Architektur hauptsächlich durch einbeziehen funktionaler parameter, nicht funktionale parameter wegen fehlender implementation schwer messbar.

\section{Ausblick}
Prozess in der Planungsphase erprobt. Implementierungsphase auch wichtig, aber nicht beschrieben. Nächste Schritte könnten sein den Prozess mit einer Implementierungsphase zu erweitern. 1 Projekt hat viele Probleme schon aufgezeigt, aber weitere, unterschiedliche Projekte wären gut um den Prozess noch zu verbessern.