\chapter{Zusammenfassung}
In der Arbeit wurde anhand eines Beispielprojektes ein UML-basierter, allgemein anwendbarer Architekturprozess erstellt, mit welchem sich reproduzierbare Softwarearchitekturen erstellen lassen. Der Architekturprozess ordent sich in die Planungsphase der Architektur ein und verwendet für die Rechtfertigung der Architekturentscheidungen einen kostenbasierten Ansatz.

Die intiale Frage - Wie kommt man von Anforderungen auf eine gute Architektur - kann wegen der ungenauen Definition einer guten Architektur nicht genau beantwortet werden. Die Frage muss anders formuliert werden, nämlich wie folgt: Wie kommt man von Anforderungen auf eine angemessene Architektur. Eine angemessene Architektur ist eine Architektur, welche die Anforderungen des/der KundIn abdeckt, nicht Mehr und nicht Weniger.

Will man die Frage beantworten, ob die in der Planungsphase erstellte Architektur angemessen ist, muss die erstellte Architektur anhand von messbaren Werten überprüft werden. Da in der Planungsphase noch keine Implementation des Systems vorliegt, ist es jedoch äußerst schwierig, bestimmte Werte wie Performance zu messen. Der Prozess begnügt sich daher mit schon messbaren, bekannten bzw. vergleichbaren Werten wie zB. den Kosten. Zusätzliche wird versucht, durch Analysen der nicht funktionalen Anforderungen der Architektur mögliche Risiken, Probleme und Parameter für die Entscheidungsfindung in der Implementationsphase zu generieren. Um die Fehlerkosten der Entscheidungen, welche auf den Analysen basieren, zu reduzieren werden architekturrelevante Parameter schon so früh wie möglich ermittelt, nämlich in der Anforderungsphase.

Eine Architektur muss jedoch nicht nur für sich selbst erstellt werden, sondern auch dem/der KundIn kommuniziert werden. Um dies zu erreichen können die kostenbasierten Entscheidungen und die UML Modelle herangezogen werden.

\section{Vorteile des erstellten Architekturprozesses}
Soll ein Architekturprozess für die Softwareentwicklung eingeführt werden, so muss dieser einen Vorteil gegenüber des Status Quo bieten. Welche Vorteile zeichnen den in der Arbeit beschriebene Architekturprozess jedoch aus?

Die Vorteile ergeben sich im Prinzip aus den treibenden Anforderungen an den Prozess und das Projekt: Der Prozess soll nachvollzieh- und reproduzierbar sein, sodass verschiedene Personen mit den gleichen Anforderungen eine identische oder zumindest stark identische Architektur erstellen. Der Fokus liegt auf der frühen Planungs- und nicht der Implementationsphase, das Projekt selbst ist ein mittelgroßes, typisches Projekt und erfordert durch dessen Anforderungen eine hohe Datensicherheit.

Daraus ergeben sich folgende Vorteile:

\begin{itemize}
  \item Gute Verständlichkeit
  \item Fokus auf Kosteneffizienz
\end{itemize}

\subsection{Gute Verständlichkeit}
Der Prozess ist aufgrund der Verwendung von UML in einer Modelliersprache dokumentiert, welche nicht nur eine große Bekanntheit und Verwendung findet, sondern auch oft im Anforderungsprozess verwendet wird. Der Architekturprozess kann damit auf dem Anforderungsprozess aufbauen und Modelle weiterentwicklen anstatt diese in komplett neue Modellierungen zu überführen. Dies hilft nicht nur bei der Kommunikation mit dem/der KundIn sondern auch mit dem Anforderungsteam: Wenn sich zB. Anforderungen ändern, können diese an den gleichen Modellen geändert werden. Die erstellten Modelle dienen zugleich auch als Dokumentation des Projektes und helfen bei der Wartung.

Zusätzlich profitiert der Prozess von den generellen Eigenschaften einer visuellen Modellierungssprache: Sie bildet das komplexe auf eine einfachere Darstellung ab und verringert die Information auf die wirklich wichtigen Bereiche. Durch die visuelle Komponente ist das Modell schneller aufnehm- und verstehbar als purer Prosatext.

Durch den Fokus auf Preis-Leistungsverhältnis ist es auch einfacher, die Probleme, Entscheidungen und Kosten dem Kunden zu kommunizieren, welcher wegen der fehlenden Vertrautheit mit dem Thema Softwarearchitektur oft die Entscheidungen und Notwendig dieses Bereiches anzweifelt \cite[S. 8-9]{softarch}. Da er mit Kostenfragen in der Regel vertraut ist, ergibt sich hier sogar eine Chance, bessere Anforderungen zu erlangen.

\subsection{Fokus auf Kosteneffizienz}
Durch den Einsatz des Prozesses in der frühen Architekturplanungsphase lassen sich hier die meisten Fehlerkosten einsparen: Nach der 10er Regel der Fehlerkosten steigen die Fehlerkosten mit dem Projektfortschritt exponentiell an. Durch die Einbeziehung von Parametern, auf welche  Architekturreviewszenarien aufbauen lassen sich auch hier im Vorfeld Kosten reduzieren: Dies betrifft nicht nur die Ermittlung der Parameter selbst, sondern auch die Möglichkeit, triviale Probleme, welche erst bei einem Architekturreview offensichtlich werden können, durch diese Anforderungen schon im Vorfeld identifizieren und beheben zu können.

Eine weitere kostensparende Eigenschaft ist, dass architekturrelevante Parameter schon in der Anforderungsphase ermittelt werden und somit die Anzahl der Kundenkontakte reduziert werden kann.

Softwarearchitekturentscheidungen sind oft ein Trade-Off: Durch konkurrierende Qualitätsanforderungen, zB. Performance und Wartbarkeit, ist es in den seltesten Fällen möglich, eine Architektur zu erstellen, welche in allen Qualitätsanforderungen brilliert. Aus diesem Grund ist es notwendig, eine bestimmte Vorgehensweise zur Erstellung, Bewertung und Entscheidung von Architekturen zu pflegen. Der Prozess bietet in dieser Hinsicht ein kostenbasiertes Entscheidungs- und Analysemodell an, mit welcher diese Entscheidungen gegenrechenbar werden. Dadurch, dass weitere Systeme nur dann erstellt werden, wenn sie einen Kostenvorteil bergen, entsteht eine kosteneffiziente und preislich angemessene Architektur.

Nicht nur tatsächliche Kosten, sondern auch Risikokosten werden im Architekturprozess mit einbezogen. Der Fokus des Prozesses liegt auf Angriffs-, Aufalls- und Änderungsszenarien, welche als Resultat des Umfeldes des Beispielprojektes gesehen werden können.

Ein weiterer Kostenvorteil ist die Möglichkeit, Anschaffungen und Architekturentscheidungen aufzuschieben, da sich der Prozess nicht auf eine physische Architektur fest legt. Die wichtigen Grundsteine, die Schnittstellen und Kommunikationswege, sind bereits geregelt, was zusätzlich eine Paralellisierung der Architekturerstellung erlaubt.


\section{Limitierungen, Probleme und Nachteile des erstellten Architekturprozesses}
Der Prozess ist zwar allgemein anwendbar, ist aber aufgrund der Herangehensweise für manche Gebiete schlechter geeignet. Die Probleme des sind vor allem in folgenden Bereiche auffindbar:

\begin{itemize}
  \item Ausgreift- und Erprobtheit
  \item Vollständigkeit
  \item Universelle Anwendbarkeit
\end{itemize}

\subsection{Ausgreift- und Erprobtheit}
Der Prozess wurde erst anhand eines einzigen Projektes erprobt und ging nicht über die Planungsphase der Architektur hinaus. Die Implementierungsphase und abschließende Überprüfung der implementierten Software wurde aus Zeit- und Umfangsgründen nicht durchgeführt. Dies führt nicht nur zu einer mangelnden Feedbackphase, in welcher Architekturprobleme während der Implementation offensichtlich werden hätten können, sondern auch zu einem zu starken Fokus auf die Domäne des Beispielprojekts.

Dies wird unter Anderem darin offensichtlich, dass der Prozess einen großen Wert auf Datensicherheit legt, welcher durch die dem Projekt zugrunde liegenden Anforderungen enstehen: Die Sicherheit und der richtige Umgang mit den Kundendaten ist eine bindende Anforderung zur Erfüllung des ISO Standards für Zertifizierungsstellen, ohne welchen das System nicht betrieben werden könnte. \cite{ISO_CERT}

Der Prozess ist somit noch unzureichend für den Einsatz in realen Projekten getestet. Vor Allem der Einsatz in unterschiedlichen Applikationsdomänen muss noch stärker geprüft werden.

\subsection{Vollständigkeit}
Die Probleme der Vollständigkeit des Prozesses sind unter Anderem eine Konsequenz der geringen Erprobtheit. Da es sehr viele unterschiedliche Risiken gibt und für unterschiedliche Projektfelder unterschiedliche Qualitäten verlangt werden, sind im Prozess nur ein Teil der möglichen Bereiche bearbeitet worden.

Außerdem behandelt der Prozess nur die Planungsphase und nicht die Architekturphase und beschäftigt sich somit weder mit der Strukturierung des Codes, noch gibt er Auskunft über die tatsächlichen eingesetzten, physischen Komponenten.


\subsection{Universelle Anwendbarkeit}
Der Architekturprozess ist vor allem geeignet für Informationssysteme, deren Fokus auf folgenden Anforderungen liegt:

\begin{itemize}
  \item Daten unterschiedlicher Vertraulichkeit müssen verarbeitet werden
  \item Datensicherheit ist wichtig
  \item Nicht jede AkteurIn kann auf alle Daten zugreifen
  \item Preis-Leistungsverhältnis ist wichtig für die Erstellung der Software
\end{itemize}

Liegt der Fokus jedoch auf anderen Bereichen, kann das Resultat der Analysen eine sehr simple Architektur sein, welche im Implementationsprozess stark angepasst werden müssen.

Ein Beispiel hierfür ist das Erstellen einer Architektur für ein sehr rechenintensives Systems, zB. für die Berechnung von Primzahlen. Hier ist es wichtig, die Berechnungen gut zu paralellisieren; der Architekturplanungsprozess würde aufgrund der Anforderungen eine einzige Softwarekomponente zur Berechnung der Primzahlen ermitteln, was zwar richtig wäre, aber wenig Hinweise zur Strukturierung der tatsächlichen physischen Systeme bieten würde.

Eine zweites Beispiel wäre ein Embedded System, welches eine starke Limitierung des Netzwerk- und Speicherdurchsatzes besitzt: Durch die Aufspaltung in mehrere Systeme, welche dann durch ein Netzwerk verbunden werden, würde die Zeit, welche zur Übermittlung der Daten aufgewendet würde stark unter der ermittelten Architektur leiden. Auch eine Virtualisierung der Komponenten auf einem System würde wegen der Speicherlimittierungen nicht in Frage kommen. Die ermittelte Architektur müsste aufgrund der Hardware Limittierungen stark überarbeitet werden.

\section{Erkenntnisse}
Bei der Erstellung des Prozesses wurden folgende Dinge ersichtlich


erstellung hat folgende erkenntnisse gebracht

Qualitätsanforderungen müssen messbar und

\subsection{Ohne messbare Parameter kein Kochrezept möglich}
\subsection{Parameter nicht zu jeder Phase messbar}
\subsection{Priorisierung von nicht funktionalen Parametern schwer möglich}
\subsection{Generische Komponentenarchitektur durch viele und schnell ändernde Kombinationen schwer möglich}
\subsection{Auf Ehrfahrungswerte kann nicht vollkommen verzichtet werden}
\subsection{Funktionale Anforderungen beeinflussen Archtitektur mehr als gedacht}
Prozess erstellt die frühe Architektur hauptsächlich durch einbeziehen funktionaler parameter, nicht funktionale parameter wegen fehlender implementation schwer messbar.

\section{Ausblick}
Der Prozess wurde in der Planungsphase eines einzigen Beispielprojektes erprobt und schließt vor dem Beginn der Implementationsphase ab. Möglichkeiten zur Verbesserung des Prozesses lassen sich nicht nur durch die Erprobung weiterer, unterschiedlicher Projekte erlangen, sondern auch eine Erweiterung des Prozesses für die Implementationsphase ist denkbar.

Eine Möglichkeit für den Ablauf der Implementationsphase wäre folgender: Zuerst werden automatisierte Testfälle für die Qualitätsanforderungen erstellt. Die Komponentenarchitektur kann dann 1:1 auf die physische Architektur umgelegt werden, für jede Komponente wird ein physisches System eingesetzt. Die Funktionalität der Komponenten wird dann durch Prototypen umgesetzt. Anhand der Ergebnisse der Tests, wird regelmäßig überprüft, ob die geforderten Werte erreicht werden. Falls sie nicht erreicht werden, kann eine andere Architektur für die jeweilige Komponente anhand eines Prototypen überprüft werden.

Die Auswahl der Architektur, welche für die jeweilige Komponenten eingesetzt wird, obliegt den Entscheidungen der ArchitektInnnen, sprich basiert auf deren Erfahrungswerten. Auch der Einsatz von Architekturreviewmethoden wie ATAM kann hier für die Entscheidungsfindung hilfreich sein. Ob die Architektur angemessen ist, kann überdies durch die bereits erstellten, automatisierten Tests überprüft werden.


