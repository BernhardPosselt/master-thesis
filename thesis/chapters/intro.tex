\chapter{Einführung}

\section{Motivation}
\subsection{Wie kommt man von Anforderungen auf eine gute Architektur}
Vermutung: Gute Architektur häng von guten Anforderungen ab, deswegen muss Anforderungsprozess erweitert werden
Probleme aufzählen die bei einer schlechten Architektur entstehen.

Vermutung: architektur hauptverantwortlich für nicht funktionale Anforderungen, also muss system unter berücksichtigung der nicht funktionalen anforderung erstellt werden

vielleicht noch genauer spezifizieren dass es keine all round solution werden soll sondern die meisten use cases abdecken soll (grafik wieviel apps SOA haben zb)

\subsection{Gibt es eine Art Kochrezept für die Architekturerstellung}
Gesucht: Ein Prozess mit dem auf die wichtigen Faktoren eingegangen werden kann.

\section{Was wird gemacht}
\subsection{Planung einer Zertifizierungsstellen Architektur}
\subsection{Modellierung mit UML}
\subsection{Anpassen der Anforderungestemplates}
\subsection{Architekturplanung Prozesserstellung, eine Art Framework}
\subsection{Dokumentation der Prozesserstellungsanläufe}

\section{Was wird nicht gemacht}
\subsection{Abdeckung des kompletten Prozesses}
Weil zu umfangreich, genauere Beschreibung im Architektur kapitel, auch erklären dass es sich heraus gestellt hat dass man nicht alles sofort planen und bewerten kann wegen fehlenden Messmöglichkeiten
\subsection{Kein Organisations- und Kommunikationsmanagement}
\subsection{Implementation}
Zu umfangreich
\subsection{Abdeckung aller möglichen Architekturfälle (Embedded, High Performance)}
\subsection{Abdeckung aller möglichen Archtekturreviewmethoden}
\subsection{Komplette Vorgaben der Bewertungs-Methoden}
Eher Hinweise auf wie man zb Ausfallkosten bewerten könnte, Methode auf eigene Anwendungsfälle abänder- und ersetzbar. Aber aufzeigen warum es gut sein kann sie dennoch schon zu behandeln
\subsection{Kompletter Anforderungsprozess}
Nur eine Erweiterung im Hinblick auf benötigte Parameter

\section{Übersicht}
Erklären was in welchen Kapiteln behandelt wird
\subsection{Modellierung}
\subsection{Anforderungen}
\subsection{Architektur}
\subsection{Prozesserstellung}
\subsection{Prozessumsetzung Anforderungen}
\subsection{Prozessumsetzung Architektur}
\subsection{Zusammenfassung}