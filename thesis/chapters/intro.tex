\chapter{Einführung}
Ein sehr junges feld, mehre modelle entstanden todo

\section{Motivation}
Nicht nur die Qualität des Sourcecodes ist für die Gesamtqualität eines Softwaresystems notwendig, sondern auch die Softwarearchitektur. Die Architektur beschreibt die Grundzüge und Pfeiler, auf welchem das System erstellt werden soll.

Eine der Grundfragen, welche die Wahl dieses Themas wesentlich beinflusst hat, ist die Frage: Wie kommt man von Anforderungen auf eine gute Architektur. Dafür muss zuerst geklärt werden, was eine gute Architektur ausmacht.

bezieht sich auf softwarequalität/erfüllung der kundenanforderungen, man muss zuerst einen blick darauf werfen was gute softwarearchitektur ausmacht -> bewertungsmethoden! wirft man einen blick auf atam, welche oft verwendet wird sieht man, bewertung anhand utility tree, vor allem nicht funktionale anforderungen, viele andere reviews bauen darauf auf, -> vermutung: nicht funktionale anforderungen liefern hauptinput bei entscheidungen -> anforderungen kommen aus anforderungsprozess -> vermutung: anforderungsprozess ist wichtig und eng verbunden mit architekturprozess

zweite motivation: es soll ein nachvollziehbarer prozess entstehen, warum prozess? prozess zitieren todo. wichtig für qualitätssicherung der arbeit, pdca. der prozess selbst soll nicht zu abstrakt sein, gut anwendbar und weil kommunikation der architektur auch zu aufgabe des architekten gehört auch gut kommunizierbar sein. Architektur hat viel mit trade offs zu tun deswegen heißt anwendbar dass nicht alles abgedeckt werden kann weil sonst zu viel. prozess selbst soll effektiv sein um kosten zu sparen weil architektur schwer rechtfertigbar ist vom kunden, deswegen ist die vermutung laut 10er regel dass prozess für planungsphase am effektivsten ist für kosten

\section{Methodik}
Um die Fragestellungen zu untersuchen und  zu beantworten, wird anhand eines  Beispielprojekts, Anforderungen und Architekturreviews versucht, eine Architektur zu erstellen. Die Probleme bei der Erstellung sollen dokumentiert werden und aus der Architektur soll ein allgemein anwendbarer Prozesses ermittelt werden.

Das Projekt selbst beschreibt das System einer Zertifizierungsstelle. Die Zertfizierungsstelle selbst ist ein mittelständischer, durchschnittlicher Betrieb, die Anforderungen sind weder zu gering noch zu exzessiv. Das System der Zertifzierungsstelle muss nicht nur Vorgaben des ISO Standards für Zertifizierungsstellen erfüllen \cite{ISO_CERT}, sondern auch Verwaltungsfunktionalitäten bereit stellen. Dazu zählen zB. die Verwaltung und Auswertung von Zertifikationsprüfungen, das Erstellen von Prüfungsterminen, für welche sich Personen anmelden können und die Verwaltung des Zahlungsverkehrs. Aufgrund des zu erfüllenden ISO Standards stehen vor allem Datenschutz und Sicherheit im Vordergrund. Da durch die Erfüllung von Qualitätsstandards, welche meist auch einen großen Wert auf Sicherheit und Datenschutz legen (zB. ISO 27001 \cite{ISO_SEC}, höhere Preise gerechtfertigt werden können, ist deren Zertifzierung aus diesem Grund jedoch auch für andere Firmen attraktiv. Außerdem sind bestimmte Regelungen gesetzlich vorgeschrieben, zB. im Datenschutzgesetz \cite[§ 14]{datenschutz}. Datenschutz und Sicherheit sind somit generell relevant und nicht einzigartig für diese Art von Projekt.

Wegen der Vermutung, dass der Anforderungsprozess eine große Rolle für den Architekturprozess spielt, sollen zusätzliche, architekturrelevante Anforderungsparameter ermittelt werden und für Analysen und Enscheidungen im Prozess verwendet werden.

Da Architekturreviews wie ATAM stark auf nicht funktionale Anforderungen setzen und die Architektur selbst diese stark beinflusst, soll versucht werden, diese als Hauptentscheidungsgrundlage für die Architekturerstellung zu verwenden. Diese Entscheidungen sollen nach einer Regel getroffen werden, um die Varianz, welche Architekturreviews wie ATAM und CBAM inne wohnen, zu reduzieren. Dies soll eine Mindestqualität der Architektur durch die Durchführung des Prozesses garantieren.

Die Architektur, welche durch den Prozess erstellt werden soll, soll kein zu kompliziertes und großes System werden. Deswegen soll die Erstellung jeder Komponente durch eine Anforderung gerechtfertigt werden.

Um die Verständlichkeit zu erhöhen sollen die Artefakte des Prozesses, falls möglich, durch UML dargestellt werden. Der Hauptfokus soll auf der Erstellung der Komponenten und Interfaces liegen.

Die Implementationsphase wird zwar nicht mehr behandelt, für sie sollen jedoch verwertbare Ausgangsparameter gefunden werden, welche Risiken aufdecken und Entscheidungshilfen für die Implementation bereit stellen.


\section{Abgrenzung}
Eine Abdeckung des kompletten Architekturprozesses ist aufgrund der Tiefe und des Umfangs nicht möglich. Würde dies versucht werden, müsste dies auf einem  höherer Abstraktionsgrad geschehen, um den Umfang nicht zu sprengen. Der Prozess selbst beschränkt sich auf die Planungsphase der Architekturerstellung. Der Anforderungs- und Implementationsprozess werden als umschließende Abschnitte zwar mit einbezogen, jedoch liegt der Fokus nur auf den Eingangs- respektive Ausgangsparametern.

Eine gute Verständlichkeit und Kommunizierbarkeit des Prozesses gehören zwar zur den erklärten Zielen, jedoch werden für diese Bereiche keine eigenen Prozesse und Vorgaben erstellt, da sich dieser Bereich stark mit Kommunikationswissenschaft und Psychologie überlappt, welche nicht im Fokus der Arbeit stehen. Auch der Anforderungsprozess überschneidet sich wesentlich mit diesen Bereichen: Er verwendet zB. bestimmte Frage- und Dokumentationstechniken, mit welchen versucht werden soll Anforderungen komplett und verständlich zu erfassen. Da der Anforderungsprozess aber die Eingabeparameter und Ziele für den Architekturprozess liefert, ist dieser für den Architekturprozess wichtig. Die Arbeit beschränkt sich deswegen auf die Beschreibung architekturrelevanter Ausgangsparameter.

Das Beispielprojekt selbst soll nur in der Architekturplanungsphase durchgeführt werden. Die Anzahl und Qualität der Anforderungen liefert genügend Parameter für die Erprobung des Prozesses, die Implementation des Systems selbst ist jedoch aufgrund der Anforderungen zu zeit- und aufwandsintensiv: Nicht nur die zu implementierenden Funktion sind hier zahlreich, sondern auch die Maßnahmen, welche für die benötigten Qualitäten - zB. Datenschutz - durchgeführt werden müssten, sind zu zahlreich.

Durch die fehlende Implementation wird daher auch auf die genaue Planung der Codestruktur verzichtet. Das bedeutet nicht nur, dass auf Patterns verzichtet wird, sondern auch, dass Architektursichten, welche sich mit der Planung des Codes beschäftigen nicht beachtet werden; dazu zählt zB. Kruchtens Development View. Grundsätzlich wäre die Planung der Module zwar möglich, aber wegen der fehlenden Implementation nicht auf ihre Korrekt- und Angemessenheit überprüfbar.

Da Architektur stark von den Anforderungen an das Systems abhängig ist, kann durch zu dominante Anforderungen ein komplett anderes Vorgehen benötigt werden. Dies trifft nicht nur auf Gebiete wie Embedded zu, wo durch die fehlende Rechenkraft und geringen Speicherplatz so effektiv wie möglich gearbeitet werden muss, sondern auch auf Systeme, welche durch schiere Größe, die Datenmenge und Rechenleistung eine eigene Aufteilung und Kommunikationswege benötigen. Auch Randbedingungen haben einen großen Einfluss auf die Architektur: Soll zB. ein System in einer sehr kurzen Zeit erstellt werden oder ist eine bestimmte Technologie wie CORBA vorgeschrieben, kann dies auch die Wahl der Komponenten beinflussen. All diese Ausnahmefälle sind hinderlich für die Erstellung eines simplen und verständlichen Prozesses. Es deshalb nicht die Absicht, einen in allen Fällen anwendbaren Prozess zu erstellen.

Die in der Arbeit verwendeten Architekturreviewmethoden werden auf ATAM und CBAM begrenzt. ATAM ist die ausgereifteste Reviewmethode \cite[S. 184]{basiswissen}, auf welcher viele andere Reviewmethoden wie ALMA und CBAM aufbauen, CBAM ist wegen des Preis-Leistungsansatzes interessant für den Prozess.

Die nach der initialen Architekturerstellung durchgeführten Analysen beanspruchen keine hundertprozentige Abdeckung aller Möglichkeiten. Auch die durchgeführten Berechnungen sind nicht bis ins kleinste Detail genau und können je nach Projekt in ihrer Genauigkeit variieren. Sie sollen vor allem auf die Analysemöglichkeiten hinweisen, welche schon vor eines klassischen Architekturreview durchführbar sind.

\section{Übersicht}
Die Arbeit teilt sich in folgende Kapitel auf:

\begin{itemize}
  \item Softwarequalität
  \item Softwarearchitektur
  \item Modellierung in der Architektur
  \item Prozesserstellungsversuche
  \item Ermittlung der Architekturanforderungen
  \item Erstellung der Architektur
  \item Zusammenfassung
\end{itemize}


\subsection{Softwarequalität}
Die Sicherstellung der Softwarequalität ist eines der wichtigsten Ziele bei der Erstellung der Softwarearchitektur, unter Anderem weil die Architektur die Struktur der zu erstellenden Software mitbestimmt und somit gewisse Qualitätsmerkmale begünstigt oder limitiert.

Um dieses Ziel zu verstehen, muss zuerst definiert werden, was Softwarequalität respektive Qualität überhaupt bedeutet. Eine Antwort darauf liefert der in ISO 9126 beschriebene Standard und dessen Nachfolger ISO 25010, auf welchen aber aus geringen Verbreitungsgründen aber nur kurz hingewiesen wird.

ISO 9126 definiert Softwarequalität in der Erfüllung der Anforderungen. Diese lassen sich in funktionale und nicht funktionale Anforderungen aufspalten, wobei bei der Architekturerstellung der primäre Fokus auf den nicht funktionalen Anforderungen liegt.

\subsection{Softwarearchitektur}
Architektur versucht durch Abstraktion die Grundpfeiler der Software fest zu legen. Dies wird standardmäßig durch die Verwendung von Architektursichten erreicht. Es gibt mehrere verschiedene Modelle, welche unterschiedliche Sichten definieren. Um zu erläutern, wie diese Sichten genau verwendet werden können, kann unter Anderem das Modell von Kruchten und das Zachman Framework verwendet werden.

Wie gestaltet sich jedoch der genaue Ablauf der Architekturerstellung? Der Architekturprozess kann in mehrere Abschnitte aufgeteilt werden: Nach einem kurzen Anfangsprozess nach der Anforderungsanalyse kann mit der Planung, Erstellung, Überprüfung der Architektur begonnen werden. Dabei muss auch die Kommunikation beachtet werden.

Für die Überprüfung der Architektur werden Architekturreviewframeworks verwendet. Das Bekannteste unter ihnen ist ATAM, welches durch mehrere Szenarientypen die Qualitätsmerkmale der gewählten Architektur zu überprüfen versucht. Eine weitere Reviewmethode ist CBAM, welche auf ATAM aufbaut aber vor Allem das Preis-Leistungsverhältnis als Entscheidungsgrundlage verwendet.

\subsection{Modellierung in der Architektur}
\subsection{Prozesserstellungsversuche}
\subsection{Ermittlung der Architekturanforderungen}
\subsection{Erstellung der Architektur}
\subsection{Zusammenfassung}