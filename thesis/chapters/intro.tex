\chapter{Einführung}
Softwarearchitektur ist ein noch ein sehr junges Feld in der Softwarebranche und oft schwer greif- und kommunizierbar \cite[S. 8]{softarch}, ist jedoch für die Qualität der erstellten Software unabdingbar. Je mehr Unternehmen ihre Datenverwaltung durch Softwarelösungen umsetzen, desto offensichtlicher werden die durch mangelhafte Architekturen verursachten Probleme. Oliver Vogel zählt unter Anderem folgende Symptome auf, welche primär auf schlechte Architekturentscheidungen zurückzuführen sind \cite[S. 7]{softarch}:

\begin{itemize}
  \item \glqq Schnittstellen, die schwer zu verwenden bzw. zu pflegen sind weil sie einen zu großen Umfang haben\grqq
  \item \glqq Quelltext, der an zahlreichen Stellen im System angepasst werden muss, wenn Systembausteine, wie beispielsweise Datenbank oder Betriebssystem, geändert werden\grqq
  \item \glqq Klassen, die sehr viele ganz unterschiedliche Verantwortlichkeiten abdecken und deshalb schwer wiederzuverwenden sind\grqq
  \item \glqq Fachklassen, deren Implementierungsdetails im gesamten System bekannt sind\grqq
\end{itemize}

Diese Probleme treten jedoch nicht nur auf der Sourcecodeebene auf: Nimmt die Größe eines Systems zu, können sich die Ressourcen einer einzelnen Komponente als unzureichend erweisen; das System muss folglich vertikal skaliert werden, sprich die Leistung muss durch das Hinzufügen weiterer Komponenten erhöht werden.

Ein weiterer Grund für die Aufspaltung eines Systems in mehrere Komponenten ist die inhärente Komplexität der einzelnen Systeme, welche Fehler und Sicherheitslücken begünstigt. Sind zB. alle wichtigen Daten auf einem System gespeichert, so können Fehler, welche auf diesem System auftreten, einen Komplettausfall des Systems verursachen. Verwaltet das Unternehmen auch oft verwendete Geschäftsdaten auf diesem System, so verursacht der Ausfall des Systems einen Arbeitsstillstand: Daten können weder eingegeben noch abgerufen werden und verursachen dem Unternehmen dadurch signifikante Kosten.

Das gleiche Problem kann auch durch die böswillige Ausnützung von Sicherheitslücken auftreten: Hat ein Angreifer erst einmal die Kontrolle über ein System, kann er auf sämtliche auf dem System gespeicherten Daten zugreifen, diese manipulieren und das System in einen unbenutzbaren Zustand versetzen. Dass dies kein unrealistisches Szenario darstellt, wird durch die in letzter Zeit veröffentlichten Sicherheitslücken wie Heartbleed\cite{heartbleed} und die Remote Execution Lücke im IIS ersichtlich\cite{iis}, welche Angreifern die Möglichkeit zur vollen Kontrolle des Systems gibt. Eine Lösung kann auch hier die Aufspaltung in mehrere Systeme darstellen, um wichtige Daten durch eine Verringerung der Angriffsfläche zu schützen.

Eine gut ausgelegte Komponentenarchitektur ist somit nicht nur für große Systeme wichtig, um eine effiziente Verarbeitung von Daten zu garantieren, sondern auch für mittlere Systeme, welche zB. bei mittelständischen Unternehmen eingesetzt werden.

\section{Forschungsleitende Fragen}
Die Grundfragen, welche diese Arbeit beantworten soll sind folgende:

\begin{itemize}
  \item Wie kommt man von Anforderungen auf eine gute Architektur?
  \item Gibt es einen allgemein gültigen, anwendbaren Architekturprozess, welcher es erlaubt, reproduzierbar Komponenten zu erstellen?
\end{itemize}

Die erste Grundfrage, welche die Wahl dieses Themas wesentlich beeinflusst hat, ist die Frage: Wie kommt man von Anforderungen auf eine gute Architektur. Anders formuliert: Wie kann eine qualitativ hochwertige Softwarearchitektur mit den verfügbaren Anforderungen erstellt werden kann.

Die zweite Grundfrage baut direkt auf der Ersten auf und dreht sich um die Erstellung von Komponenten durch einen definierten Prozess. Was genau ist jedoch ein Prozess? Das Merriam-Webster Lexikon definiert einen Prozess als eine Abfolge von Handlungen, die entweder etwas produzieren oder zu einem bestimmten Resultat führen\cite{processweb}. Diese Handlungen können nicht nur qualitätssichernde Schritte beinhaltet, um einen gewissen Qualitätsstandard bei der Architekturerstellung zu garantieren, sondern auch eine Effizienzsteigerung ermöglichen.

Der Prozess soll allgemein anwendbar und reproduzierbar sein, kann also nicht wie die meisten Architekturprozessbeschreibungen auf einer zu abstrakten Ebene definiert werden.

\section{Methodik}
Um die Fragestellungen zu untersuchen und  zu beantworten, wird anhand eines  Beispielprojekts, Anforderungen und Architekturreviews versucht, eine Architektur zu erstellen. Die Probleme bei der Erstellung sollen dokumentiert werden und aus der Architektur soll ein allgemein anwendbarer Prozess ermittelt werden.

Das Projekt dreht sich um ein System einer Personenzertifizierungsstelle. Die Zertifizierungsstelle selbst ist ein mittelständischer, durchschnittlicher Betrieb. Das System der Zertifzierungsstelle muss die Vorgaben des ISO Standards für Personenzertifizierungsstellen erfüllen\cite{ISO_CERT} und stellt diverse Verwaltungsfunktionalitäten bereit. Dazu zählen zB. die Verwaltung und Auswertung von Zertifizierungssprüfungen, das Erstellen von Prüfungsterminen, für welche sich Personen anmelden können und die Verwaltung des Zahlungsverkehrs.

Durch die Rahmenbedinungen des einzuhaltenden ISO Standards stehen vor allem Datenschutz und Sicherheit im Vordergrund. Der Hohe Fokus auf Datenschutz und Sicherheit mag zwar eine Eigenheit des Projekts sein, ist aber durchaus relevant und attraktiv für Projekte in anderen Bereichen.

Regelungen bezüglich Datenschutz sind zB. fest im österreichischen Gesetz für Datenschutz verankert. Das Gesetz schreibt vor, dass \glqq die Zugriffsberechtigung auf Daten und Programme und der Schutz der Datenträger vor der Einsicht und Verwendung durch Unbefugte\grqq \cite[§ 14, 5]{datenschutz} geregelt werden müssen, sowie \glqq die Berechtigung zum Betrieb der Datenverarbeitungsgeräte\grqq \ festgelegt \glqq und jedes Gerät durch Vorkehrungen bei den eingesetzten Maschinen oder Programmen gegen die unbefugte Inbetriebnahme\grqq \cite[§ 14, 6]{datenschutz} abgesichert werden muss. Somit ist grundsätzlich jedes Softwareprojekt, welches in Österreich eingesetzt wird und personenbezogene Daten verwaltet zur Einhaltung dieser gesetzlichen Rahmenbedingungen verpflichtet.

Auch freiwillige Zertifizierungen im Bezug auf Datenschutz und Sicherheit, wie zB. wie ISO 27001 \cite{ISO_SEC}, können sich für Unternehmen rechnen. So kann zB. mit Gütesiegeln und Zertifzierungen geworben und somit der höhere Preis gegenüber konkurrierenden Firmen gerechtfertigt werden. Dies ist besonders relevant, da zB. konkurrierenden Firmen aus dem Ausland durch niedrigere Löhne und bessere Voraussetzungen Angebote österreichischer Firmen weit unterbieten können.

Aufgrund der Vermutung, dass der Anforderungsprozess eine große Rolle für den Architekturprozess spielt, sollen zusätzliche, architekturrelevante Anforderungsparameter ermittelt und für weitere Analysen und Entscheidungen im Prozess verwendet werden.

Da Architekturreviews wie ATAM stark auf nicht funktionale Anforderungen setzen und die Architektur selbst diese stark beeinflusst, soll versucht werden, diese als Hauptentscheidungsgrundlage für die Architekturerstellung zu verwenden. Diese Entscheidungen sollen nach einem festgelegten Regelwerk getroffen werden, um die Varianz, welche Architekturreviews wie ATAM und CBAM inne wohnen, zu reduzieren. Dies soll eine Mindestqualität der Architektur nach der Durchführung des Prozesses garantieren.

Die Architektur, welche durch den Prozess erstellt werden soll, soll kein zu kompliziertes und großes System werden. Deswegen soll die Erstellung jeder Komponente durch eine Anforderung gerechtfertigt werden können.

Um die Verständlichkeit zu erhöhen sollen die Artefakte des Prozesses, falls möglich, durch UML dargestellt werden. Der Hauptfokus soll auf der Erstellung der Komponenten und Interfaces liegen.

Die Implementationsphase wird zwar nicht mehr behandelt, für sie sollen jedoch verwertbare Ausgangsparameter gefunden werden, welche Risiken aufdecken und Entscheidungshilfen für die Implementation bereit stellen.


\section{Abgrenzung}
Eine Abdeckung des kompletten Architekturprozesses ist aufgrund der Tiefe und des Umfangs nicht möglich. Würde dies versucht werden, müsste dies zudem auf einem  höherer Abstraktionsgrad geschehen, was wiederum die Anwendbarkeit des Prozesses erschwert. Der Prozess selbst beschränkt sich deswegen auf die Planungsphase der Architekturerstellung. Der Anforderungs- und Implementationsprozess werden als umschließende Abschnitte zwar mit einbezogen, jedoch liegt der Fokus nur auf den Eingangs- respektive Ausgangsparametern.

Eine gute Verständlichkeit und Kommunizierbarkeit des Prozesses gehören zwar zu den erklärten Zielen, jedoch werden für diese Bereiche keine eigenen Richtlinien und Vorgaben erstellt. Diese Bereiche überschneiden sich stark mit den Feldern der Kommunikationswissenschaft und Psychologie, welche nicht im Fokus dieser Arbeit stehen. Auch der Anforderungsprozess überschneidet sich wesentlich mit diesen Bereichen und wird daher nicht ganzheitlich beschrieben: Er verwendet zB. bestimmte Frage- und Dokumentationstechniken, mit welchen versucht werden soll, Anforderungen komplett, korrekt und verständlich zu erfassen. Da der Anforderungsprozess aber die Eingabeparameter und Ziele für den Architekturprozess liefert, ist dieser für den Architekturprozess wichtig. Die Arbeit beschränkt sich deswegen auf die Beschreibung architekturrelevanter Ausgangsparameter, welche im Anforderungsprozess ermittelt werden.

Das Beispielprojekt selbst soll nur in der Architekturplanungsphase durchgeführt werden. Die Anzahl und Qualität der Anforderungen liefert genügend Parameter für die Erprobung des Prozesses, die Implementation des Systems selbst ist jedoch aufgrund der Anforderungen zu zeit- und aufwandsintensiv: Nicht nur die zu implementierenden Funktionen sind hier zu zahlreich, sondern auch die Maßnahmen, welche für die benötigten Qualitäten - zB. Datenschutz - durchgeführt werden müssten.

Durch die fehlende Implementation wird daher auch auf die genaue Planung der Codestruktur verzichtet. Das bedeutet nicht nur, dass auf Patterns verzichtet wird, sondern auch, dass Architektursichten, welche sich mit der Planung des Codes beschäftigen nicht beachtet werden; dazu zählt zB. Kruchtens Development View. Grundsätzlich wäre die Planung der Module zwar möglich, aber wegen der fehlenden Implementation nicht auf ihre Korrekt- und Angemessenheit überprüfbar.

Da Architektur stark von den Anforderungen an das Systems abhängig ist, kann durch zu dominante Anforderungen ein komplett anderes Vorgehen benötigt werden. Dies trifft nicht nur auf Gebiete wie Embedded zu, wo durch die fehlende Rechenkraft und geringen Speicherplatz so effektiv wie möglich gearbeitet werden muss, sondern auch auf Systeme, welche durch ihre Größe, Datenmengen und Rechenleistungen eine eigene Aufteilung und Planung benötigen.

Auch Rahmenbedingungen haben einen großen Einfluss auf die Architektur: Soll zB. ein System in einer sehr kurzen Zeit erstellt werden oder ist eine bestimmte Technologie wie CORBA vorgeschrieben, kann dies auch die Wahl der Komponenten beeinflussen. All diese Ausnahmefälle sind hinderlich für die Erstellung eines simplen und verständlichen Prozesses. Deswegen wird bei der Erstellung auf einen kompletten, auf jede Ausnahme anwendbaren Prozess verzichtet.

Die in der Arbeit verwendeten Architekturreviewmethoden werden auf ATAM und CBAM begrenzt. ATAM ist die ausgereifteste Reviewmethode \cite[S. 184]{basiswissen}, auf welcher viele andere Reviewmethoden wie ALMA und CBAM aufbauen. CBAM ist wegen des Preis-Leistungsansatzes interessant für den Prozess.

Die nach der initialen Architekturerstellung durchgeführten Analysen beanspruchen keine hundertprozentige Abdeckung aller Möglichkeiten. Auch die durchgeführten Berechnungen sind nicht bis ins kleinste Detail genau und können je nach Projekt in ihrer Genauigkeit variieren. Sie sollen vor allem auf die Analysemöglichkeiten hinweisen, welche schon vor einem  Architekturreview durchführbar sind.

\section{Übersicht}
Die Arbeit teilt sich in folgende Kapitel auf:

\begin{itemize}
  \item Softwarequalität
  \item Softwarearchitektur
  \item Modellierung in der Architektur
  \item Prozesserstellungsversuche
  \item Ermittlung der Architekturanforderungen
  \item Erstellung der Architektur
  \item Zusammenfassung
\end{itemize}


\subsection{Softwarequalität}
Die Sicherstellung der Softwarequalität ist eines der wichtigsten Ziele bei der Erstellung der Softwarearchitektur, unter Anderem weil die Architektur die Struktur der zu erstellenden Software mitbestimmt und somit gewisse Qualitätsmerkmale begünstigt oder limitiert.

Um dieses Ziel zu verstehen, muss zuerst definiert werden, was Softwarequalität respektive Qualität überhaupt bedeutet. Eine Antwort darauf liefert der in ISO 9126 beschriebene Standard und dessen Nachfolger ISO 25010, auf welchen aber aus geringen Verbreitungsgründen nur kurz hingewiesen wird.

ISO 9126 definiert Softwarequalität in der Erfüllung der Anforderungen. Diese lassen sich in funktionale und nicht funktionale Anforderungen aufspalten, wobei bei der Architekturerstellung der primäre Fokus auf den nicht funktionalen Anforderungen liegt.

\subsection{Softwarearchitektur}
Architektur versucht durch Abstraktion eine Übersicht über das zu erstellende System zu gewähren und legt die Grundpfeiler der Software fest. Dies wird standardmäßig durch die Verwendung von Architektursichten erreicht. Es gibt mehrere verschiedene Modelle, welche unterschiedliche Sichten definieren. Um zu erläutern, wie diese Sichten genau verwendet werden können, kann unter Anderem das Modell von Kruchten und das Zachman Framework herangezogen werden.

Wie gestaltet sich jedoch der genaue Ablauf der Architekturerstellung? Der Architekturprozess kann in mehrere Abschnitte aufgeteilt werden: Nach einem kurzen Anfangsprozess nach der Anforderungsanalyse kann mit der Planung, Erstellung, Überprüfung der Architektur begonnen werden. Dabei muss auch die Kommunikation beachtet werden.

Für die Überprüfung der Architektur werden Architekturreviewframeworks verwendet. Das Bekannteste unter ihnen ist ATAM, welches durch mehrere Szenarientypen die Qualitätsmerkmale der gewählten Architektur zu überprüfen versucht. Eine weitere Reviewmethode ist CBAM, welche auf ATAM aufbaut aber vor Allem das Preis-Leistungsverhältnis als Entscheidungsgrundlage verwendet.

\subsection{Modellierung in der Architektur}
Modelle besitzen ein verkürzende Wirkung und bieten sich damit vor Allem zur Dokumentation und Kommunikation der Architektur und Anforderungen an. Die in ISO 42010 beschriebenen Architektursichten verlangen zudem ein oder mehrere Architekturmodelle. Diese Modelle können mit Hilfe von auf die Architektur zugeschnittenen Modellierungssprachen umgesetzt werden.

Hier bietet wegen der großen Verbreitung vor allem UML an. UML bietet alle für die Architektur nötigen Modelle. Es werden jedoch nicht alle Modelle benötigt. Für die Arbeit werden lediglich das Usecase-, Komponenten-, Klassen- und Aktivitätsdiagramm verwendet. Zusätzlich wird ein Kontextdiagramm eingesetzt, welches jedoch nicht Teil von UML ist.

\subsection{Prozesserstellungsversuche}
Anhand der priorisierten nicht funktionale Anforderungen wurde nun versucht, einen Architekturprozess abzuleiten. Dies scheiterte jedoch aufgrund der Anforderung an den Prozess: Der Prozess soll allgemein anwendbar und reproduzierbar sein; die Bewertung der nicht funktionalen Anforderungen war aber in der Planungsphase durch die fehlende Implementation zu ungenau. Auch durch die Verwendung von Kohäsionswerten ließ sich keine eindeutige Architekturerstellung ableiten und die Architekturen wurden zu teuer.

Erst die Aufspaltung in Daten und Zugriffe durch AkteurInnen verhalf zu einer reproduzierbaren Architektur, welche groß genug für Bewertungen war und nicht zu teuer war.

\subsection{Ermittlung der Architekturanforderungen}
\subsection{Erstellung der Architektur}
\subsection{Zusammenfassung}