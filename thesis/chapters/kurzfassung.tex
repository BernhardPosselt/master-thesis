Softwarearchitektur ist ein sehr breites Gebiet und sehr abstraktes Gebiet der Softwareentwicklung, ist aber Voraussetzung für qualitativ hochwertige Software. Der hohe Abstraktionsgrad ist aber oft ein Hindernis für eine dem/der KundIn kommunizierbare, reproduzierbare und qualitativ hochwertige Architektur.

Deswegen wird in dieser Arbeit ein reproduzierbarer und kommunizierbarer Architekturerstellungsprozess entworfen. Da laut Zehner-Regel der Fehlerkosten die Behebungskosten unentdeckter Fehler im Laufe des Projektes exponentiell ansteigen, beschränkt sich der Prozess auf die frühe Architekturplanungsphase, um eine möglichst große Wirkung zu erzielen.

Nach mehreren Versuchen, welche wie viele Architekturreviewmethoden stark auf die Priorisierung der Qualitätsmerkmale der Anforderungen aufbauten, wurde jedoch klar, dass sich in der frühen Planungsphase aufgrund der fehlenden Implementation zu wenig messbare Werte für einen reproduzierbaren Prozess ermitteln lassen.

Der Prozess baut daher auf verfügbaren Werten wie Nachbarsystemen, Netzwerken und Risikokosten im Bezug auf unerlaubten Datenzugriff auf. Zusätzlich versucht er durch Analysen der ermittelten Komponentenstruktur Hinweise auf in der Implementationsphase problematische Bereiche zu geben.

Der schließlich erstellte Prozess ist reproduzierbar und die resultierende Architektur aufgrund der Kosten und verwendeten Modellierungssprache UML gut dem/der KundIn kommunizierbar.