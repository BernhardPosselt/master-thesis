Software architecture is a very wide and abstract subject in the field of software development but is required to create high quality software. Because of the high grade of abstraction, communicating and creating a reproducible, high quality software architecture is often very hard.

Therefore this thesis will outline a reproducible and communicable software architecture process. Because errors are more costly to fix in a later phase of software development the process itself will focus only on the planing phase in order to achieve high level of effectiveness.

After various attempts to create a software architecture process based on prioritized quality factors which in turn are also often used in many software review processes, it became obvious that there were too few measurable values to create a reproducible process. This was mostly a result of the missing implementation.

The process therefore builds on early available information such as adjacent systems, networks and costs which would be created through unauthorized data access. Additionally the process analyses the created component structure to reveal possible risk areas which should be closely monitored in the following implementation phase.

The result is a reproducible and, because of its cost based approach and modeling language UML, can be communicated well to the client.