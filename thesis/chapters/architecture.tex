\chapter{Softwarearchitektur}
Softwarearchitektur beschäftigt sich mit dem Erstellen des Fundamentes eines Softwareprojektes \cite[S. 9]{softarch}. Der Fokus liegt daher nicht auf Implementationsdetails sondern auf Schnittstellen, welche das System und die Zusammenarbeit der Teilkomponenten beschreiben \cite[S. 8]{basiswissen}. Ziel der Softwarearchitektur ist es, durch Abstraktion der Details einen Überblick über die zu implementierenden Komponenten zu gewinnnen, welcher es ermöglicht, frühe Entscheidungen über die Probleme und Angemessenheit des Systems zu erlangen und somit Risiken, Fehler- und Änderungskosten zu vermeiden oder zu verringern \cite[S. 10]{softarch}.

Der Detailgrad der Schnittstellen hängt wesentlich von der Größe des Projektes ab \cite[S. 7-12]{basiswissen}. Um die Komplexität und Abstraktion zu beherrschen, können Architektursichten verwendet werden, welche einen Überblick über Teilbereiche des Systems ermöglichen \cite{ISO_ARCH}. Ein Beispiel dafür ist Kruchtens 4+1 Sichtenmodell \cite{kruch}, welches folgende Sichten beschreibt:

\begin{itemize}
  \item Logical View: Logische Sicht auf das System, welches die Klassen und deren Beziehungen untereinander darstellt
  \item Process View: Beschäftigt sich mit nicht funktionalen Anforderungen wie Performance und Fehlertolleranz und deren Implementierung
  \item Development View: Beschreibt die Aufteilung des Systems in Module und Bibliotheken
  \item Physical View: Regelt und beschreibt die Verteilung der Komponenten auf physische Systeme
  \item Scenarios: Usecases, welche die Nutzung des Systems beschreiben
\end{itemize}

Kruchten verwendet eigene Diagramme für jede Sicht, die an UML angelehnt aber zum Teil verschieden sind, etablierte sich UML doch erst zwei Jahre später nach der Veröffentlichung seines Modelles \cite{glasklar}[S. 5].

Zeitlich ordnet sich die Architekturphase zwischen dem Ermitteln der Anforderungen und der tatsächlichen Implementation ein. Da die Erstellung von Software heutzutage mehr ein agiler und inkrementeller Prozess ist, wird auch die Erstellung und die Mitwirkung der Softwarearchitektur in diese sich wiederholenden Prozesse mit einbezogen. \cite[S. 7]{basiswissen}


\section{Was ist eine gute Architektur}
Zitate; Architektur ist ein abwägen von Vor- und Nachteilen. Es gibt keine perfekte Architektur, jede Entscheidung ist mit Vor und Nachteilen versehen.

\section{Architekturprozess Lifecycle}
Diagramm wie der Architekturprozess im groben abläuft. Dann den Teil markieren auf dem der Fokus der Arbeit liegt

\section{Architekturbewertungsmethoden}
Kurze Einführung in Architekturbewertungsmethoden

\subsection{ATAM}
\subsection{CBAM}