\documentclass[Master,MSE,german]{twbook}
\usepackage[ansinew]{inputenc}
\usepackage[T1]{fontenc}
\newcommand{\FHTWCitationType}{IEEE}
\usepackage{bibgerm}

% Definition Code-Listings Formatierung:
\usepackage[final]{listings}
\lstset{captionpos=b, numberbychapter=false,caption=\lstname,frame=single, numbers=left, stepnumber=1, numbersep=2pt, xleftmargin=15pt, framexleftmargin=15pt, numberstyle=\tiny, tabsize=3, columns=fixed, basicstyle={\fontfamily{pcr}\selectfont\footnotesize}, keywordstyle=\bfseries, commentstyle={\color[gray]{0.33}\itshape}, stringstyle=\color[gray]{0.25}, breaklines, breakatwhitespace, breakautoindent}
\lstloadlanguages{[ANSI]C, C++, [gnu]make, gnuplot, Matlab}

\makeatletter
\renewcommand\lstlistingname{Quellcode}
\renewcommand\lstlistlistingname{Quellcodeverzeichnis}

% Definition des Macros listoflolentryname analog zu listoflofentryname und listoflotentryname der KOMA-Klasse
\newcommand\listoflolentryname\lstlistingname
% Neudefinition der Zeilen des Quellcodeverzeichnisses wenn die Option listof=entryprefix gew�hlt wurde
\@ifclasswith{scrbook}{listof=entryprefix}
{%
    \renewcommand\l@lstlisting[2]{\@dottedtocline{1}{1.5em}{1em}{\listoflolentryname~#1}{#2}}
}{%
}
\makeatother
\newcommand{\listofcode}{\phantomsection\lstlistoflistings}

%
% Eintr�ge f�r Deckblatt, Kurzfassung, etc.
%
\title{Architekturerstellung auf Basis von Anforderungen}
\author{Bernhard Posselt, BSc}
\studentnumber{1310299032}
\supervisor{Mag. Maria-Therese Teichmann}
\place{Wien}
\kurzfassung{sehr kurz}
\schlagworte{Schlagwort1, Schlagwort2, Schlagwort3, Schlagwort4}
\outline{abstract da}
\keywords{Keyword1, Keyword2, Keyword3, Keyword4}

\begin{document}


\maketitle

%
% .. und hier beginnt die eigentliche Arbeit. Viel Erfolg beim Verfassen!
%

* Einf�hrung
 * Motivation
  * Wie kommt man von Anforderungen auf eine gute Architektur
    * Vermutung: Gute Architektur h�ng von guten nicht funktionalen Anforderungen ab
  * Wie kann ich im Planungsprozess schon Fehlerkosten vermeiden
  * Gibt es eine Art Kochrezept um von fachlichen auf technische Anforderungen zu kommen
 * Was wird gemacht
  * Planung einer Zertifizierungsstellen Architektur
  * Modellierung mit UML
  * Anpassen der Anforderungestemplates
  * Architekturplanung Prozesserstellung, eine Art Framework
  * Dokumentation der Prozesserstellungsanl�ufe
 * Was wird nicht gemacht
  * Abdeckung des kompletten Prozesses
  * Kein Organisations- und Kommunikationsmanagement
  * Implementation
  * Abdeckung aller m�glichen Architekturf�lle (Embedded, High Performance)
  * Komplette Vorgaben der Sch�tz-Methoden
 * �bersicht
  * Modellierung
  * Anforderungen
  * Architektur
  * Prozesserstellung
  * Prozessumsetzung Anforderungen
  * Prozessumsetzung Architektur
  * Zusammenfassung
* Modellierung
 * Was ist UML
 * Diagramme
  * Kontextdiagramm
  * Komponentendiagram
  * Klassendiagramm
  * Aktivit�tsdiagramm
  * Usecasediagramm
* Anforderungen (ISO 9126)
 * Funktionale Anforderungen
  * Suitability
  * Security
  * Accuracy
  * Interoperability
  * Functional Compliance
 * Nicht funktionale Anforderungen
  * Reliability
  * Usability
  * Efficiency
  * Maintainability
  * Portability
 * Arhitekturrelevante Anforderungen
  * Messbarkeit
  * Planung
  * Umsetzung
 * Nicht architekturrelevante Anforderungen
 * Anforderungstemplate
* Architektur
 * Was ist eine gute Architektur
 * Architekturprozess Lifecycle
 * Architekturbewertungsmethoden
  * ATAM
  * CBAM
* Prozesserstellung
 * Artefakte
  * Wer (Akteure)
  * Was (Daten)
  * Warum (Anforderungen/Usecases)
  * Womit (Komponenten)
 * Vom Usecase zur Komponente
 * Von einer gesch�tzten Architektur auf eine verfeinerte Architektur durch nicht funktionale Anforderungen
 * Von den Daten zur Architektur
 * Von den Daten und den Akteuren zur Architektur
* Prozess Anforderungen
 * Modellierung der Usecases
 * �berf�hren der Usecases in Anforderungstemplate
 * Anpassen des Anforderungstemplates
  * Wachstumsszenarien und Wahrscheinlichkeit
  * �nderungsszenarien
  * Businesskritikalit�t
   * Ausfallkosten
   * Datenverlustszenarien
  * Rahmenbedingungen
 * Modellierung der Daten
 * Modellierung der Akteure und Partnersysteme
 * Modellierung der Daten x Akteure (CRUD Matrix)

* Prozess Architekturplanung
 * Erstellen der Minimalen Architektur
 * Datenaufteilung nach Vertrautheitslevel
 * Akteuraufteilung nach Vertrautheitslevel
 * Erstellen der Datenminimalarchitektur
  * Datenfl�sse
  * Erstellen von Daten-Gateways/Firewalls
  *
 * Einbeziehen der Akteure
  * Trennen von Akteursystemen basierend auf Daten und Vertrautheitslevel
 * Modellieren der Komponenten Interfaces (Klassen Diagramm)
 * Bewertung der Architektur
  * Single Point of Failure Analyse (Matrix Komponente x Usecase)
  * Ausfallskosten (inkl. Wachstumsszenarien)
  * Angriffsszenarien
 * Changemanagement
  * �nderungen der Daten
  * �nderungen der Akteure
  * �nderungen der Usecases
  * �nderungen der nicht funktionalen Anforderungen

* Zusammenfassung
 * Warum ist der Prozess gut
  * 10er Regel der Fehlerkosten
  * L�sst Entscheidungen aufschieben
 * Anwendungsf�lle
 * Limitierungen
  * �berpr�fbarkeit der Anforderungen
  * Akteure/Daten �ndern sich oft
  * Erfahrungswerte
  * Plangungsphase
  * Implementierungsphase
  * Extremarchitekturen


\chapter{Einleitung}
a \cite{Ko05a}[S. 1]
\section{blug}
b
\chapter{Grundlagen}
c
\chapter{Zusammenfassung}
d
\clearpage
\bibliographystyle{gerabbrv}
\bibliography{Literatur}
\clearpage

% Das Abbildungsverzeichnis
\listoffigures
\clearpage

% Das Tabellenverzeichnis
\listoftables
\clearpage

% Das Quellcodeverzeichnis
\listofcode
\clearpage

\chapter{Abk�rzungsverzeichnis}
\begin{acronym}[XXXXX]
    \acro{ABC}[ABC]{Alphabet}
    \acro{WWW}[WWW]{world wide web}
    \acro{ROFL}[ROFL]{Rolling on floor laughing}
\end{acronym}
\end{document}
