\documentclass[Master,MSE,german]{twbook}
\usepackage[utf8]{inputenc}
\usepackage[T1]{fontenc}
\newcommand{\FHTWCitationType}{IEEE}
\usepackage{bibgerm}
\usepackage{float}
\usepackage{eurosym}

% Definition Code-Listings Formatierung:
\usepackage[final]{listings}
\lstset{captionpos=b, numberbychapter=false,caption=\lstname,frame=single, numbers=left, stepnumber=1, numbersep=2pt, xleftmargin=15pt, framexleftmargin=15pt, numberstyle=\tiny, tabsize=3, columns=fixed, basicstyle={\fontfamily{pcr}\selectfont\footnotesize}, keywordstyle=\bfseries, commentstyle={\color[gray]{0.33}\itshape}, stringstyle=\color[gray]{0.25}, breaklines, breakatwhitespace, breakautoindent}
\lstloadlanguages{[ANSI]C, C++, [gnu]make, gnuplot, Matlab}

\makeatletter
\renewcommand\lstlistingname{Quellcode}
\renewcommand\lstlistlistingname{Quellcodeverzeichnis}

% Definition des Macros listoflolentryname analog zu listoflofentryname und listoflotentryname der KOMA-Klasse
\newcommand\listoflolentryname\lstlistingname
% Neudefinition der Zeilen des Quellcodeverzeichnisses wenn die Option listof=entryprefix gewählt wurde
\@ifclasswith{scrbook}{listof=entryprefix}
{%
    \renewcommand\l@lstlisting[2]{\@dottedtocline{1}{1.5em}{1em}{\listoflolentryname~#1}{#2}}
}{%
}
\makeatother
\newcommand{\listofcode}{\phantomsection\lstlistoflistings}

%
% Einträge für Deckblatt, Kurzfassung, etc.
%
\title{Erstellung eines Prozesses für die Planungsphase der Softwarearchitektur}
\author{Bernhard Posselt, BSc}
\studentnumber{1310299032}
\supervisor{Mag. Maria-Therese Teichmann}
\place{Wien}
\kurzfassung{Softwarearchitektur ist ein sehr breites Gebiet und sehr abstraktes Gebiet der Softwareentwicklung, ist aber Voraussetzung für qualitativ hochwertige Software. Der hohe Abstraktionsgrad ist aber oft ein Hindernis für eine dem/der KundIn kommunizierbare, reproduzierbare und qualitativ hochwertige Architektur.

Deswegen wird in dieser Arbeit ein reproduzierbarer und kommunizierbarer Architekturerstellungsprozess entworfen. Da laut Zehner-Regel der Fehlerkosten die Behebungskosten unentdeckter Fehler im Laufe des Projektes exponentiell ansteigen, beschränkt sich der Prozess auf die frühe Architekturplanungsphase, um eine möglichst große Wirkung zu erzielen.

Nach mehreren Versuchen, welche wie viele Architekturreviewmethoden stark auf die Priorisierung der Qualitätsmerkmale der Anforderungen aufbauten, wurde  klar, dass sich in der frühen Planungsphase aufgrund der fehlenden Implementation zu wenig messbare Werte für einen reproduzierbaren Prozess ermitteln lassen.

Der Prozess baut daher auf verfügbaren Werten wie Nachbarsystemen, Netzwerken und Risikokosten im Bezug auf unerlaubten Datenzugriff auf. Zusätzlich versucht er durch Analysen der ermittelten Komponentenstruktur Hinweise auf in der Implementationsphase problematische Bereiche zu geben.

Der schließlich erstellte Prozess ist reproduzierbar und die resultierende Architektur aufgrund der Kosten und verwendeten Modellierungssprache UML gut dem/der KundIn kommunizierbar.}
\schlagworte{Schlagwort1, Schlagwort2, Schlagwort3, Schlagwort4}
\outline{Software architecture is a very wide and abstract subject in the field of software development but is required to create high quality software. Because of the high grade of abstraction, communicating and creating a reproducible, high quality software architecture is often very hard.

Therefore this thesis will outline a reproducible and communicable software architecture process. Because errors are more costly to fix in a later phase of software development the process itself will focus only on the planing phase in order to achieve high level of effectiveness.

After various attempts to create a software architecture process based on prioritized quality factors which in turn are also often used in many software review processes, it became obvious that there were too few measurable values to create a reproducible process. This was mostly a result of the missing implementation.

The process therefore builds on early available information such as adjacent systems, networks and costs which would be created through unauthorized data access. Additionally the process analyses the created component structure to reveal possible risk areas which should be closely monitored in the following implementation phase.

The result is a reproducible and, because of its cost based approach and modeling language UML, can be communicated well to the client.}
\keywords{Keyword1, Keyword2, Keyword3, Keyword4}

\begin{document}


\maketitle

\chapter{Einführung}

\section{Motivation}


\subsection{Wie kommt man von Anforderungen auf eine gute Architektur}
Vermutung: Gute Architektur häng von guten Anforderungen ab, deswegen muss Anforderungsprozess erweitert werden
Probleme aufzählen die bei einer schlechten Architektur entstehen.

Vermutung: architektur hauptverantwortlich für nicht funktionale Anforderungen, also muss system unter berücksichtigung der nicht funktionalen anforderung erstellt werden

vielleicht noch genauer spezifizieren dass es keine all round solution werden soll sondern die meisten use cases abdecken soll (grafik wieviel apps SOA haben zb)

\subsection{Gibt es eine Art Kochrezept für die Architekturerstellung}
Gesucht: Ein Prozess mit dem auf die wichtigen Faktoren eingegangen werden kann.

\section{Was wird gemacht}
\subsection{Planung einer Zertifizierungsstellen Architektur}
\subsection{Modellierung mit UML}
\subsection{Anpassen der Anforderungestemplates}
\subsection{Architekturplanung Prozesserstellung, eine Art Framework}
\subsection{Dokumentation der Prozesserstellungsanläufe}

\section{Was wird nicht gemacht}
\subsection{Abdeckung des kompletten Prozesses}
Weil zu umfangreich, genauere Beschreibung im Architektur kapitel, auch erklären dass es sich heraus gestellt hat dass man nicht alles sofort planen und bewerten kann wegen fehlenden Messmöglichkeiten
\subsection{Kein Organisations- und Kommunikationsmanagement}
\subsection{Implementation}
Zu umfangreich
\subsection{Abdeckung aller möglichen Architekturfälle (Embedded, High Performance)}
\subsection{Abdeckung aller möglichen Archtekturreviewmethoden}
\subsection{Komplette Vorgaben der Bewertungs-Methoden}
Eher Hinweise auf wie man zb Ausfallkosten bewerten könnte, Methode auf eigene Anwendungsfälle abänder- und ersetzbar. Aber aufzeigen warum es gut sein kann sie dennoch schon zu behandeln
\subsection{Kompletter Anforderungsprozess}
Nur eine Erweiterung im Hinblick auf benötigte Parameter

\section{Übersicht}
Die Arbeit teilt sich in folgende Kapitel auf:

\begin{itemize}
  \item Softwarequalität
  \item Softwarearchitektur
  \item Modellierung in der Architektur
  \item Prozesserstellungsversuche
  \item Ermittlung der Architekturanforderungen
  \item Erstellung der Architektur
  \item Zusammenfassung
\end{itemize}


\subsection{Softwarequalität}
Die Sicherstellung der Softwarequalität ist eines der wichtigsten Ziele bei der Erstellung der Softwarearchitektur, unter Anderem weil die Architektur die Struktur der zu erstellenden Software mitbestimmt und somit gewisse Qualitätsmerkmale begünstigt oder limitiert.

Um dieses Ziel zu verstehen, muss zuerst definiert werden, was Softwarequalität respektive Qualität überhaupt bedeutet. Eine Antwort darauf liefert der in ISO 9126 beschriebene Standard und dessen Nachfolger ISO 25010, auf welchen aber aus geringen Verbreitungsgründen aber nur kurz hingewiesen wird.

ISO 9126 definiert Softwarequalität in der Erfüllung der Anforderungen. Diese lassen sich in funktionale und nicht funktionale Anforderungen aufspalten, wobei bei der Architekturerstellung der primäre Fokus auf den nicht funktionalen Anforderungen liegt.

\subsection{Softwarearchitektur}
Architektur versucht durch Abstraktion die Grundpfeiler der Software fest zu legen. Dies wird standardmäßig durch die Verwendung von Architektursichten erreicht. Es gibt mehrere verschiedene Modelle, welche unterschiedliche Sichten definieren. Um zu erläutern, wie diese Sichten genau verwendet werden können, kann unter Anderem das Modell von Kruchten und das Zachman Framework verwendet werden.

Wie gestaltet sich jedoch der genaue Ablauf der Architekturerstellung? Der Architekturprozess kann in mehrere Abschnitte aufgeteilt werden: Nach einem kurzen Anfangsprozess nach der Anforderungsanalyse kann mit der Planung, Erstellung, Überprüfung der Architektur begonnen werden. Dabei muss auch die Kommunikation beachtet werden.

Für die Überprüfung der Architektur werden Architekturreviewframeworks verwendet. Das Bekannteste unter ihnen ist ATAM, welches durch mehrere Szenarientypen die Qualitätsmerkmale der gewählten Architektur zu überprüfen versucht. Eine weitere Reviewmethode ist CBAM, welche auf ATAM aufbaut aber vor Allem das Preis-Leistungsverhältnis als Entscheidungsgrundlage verwendet.

\subsection{Modellierung in der Architektur}
\subsection{Prozesserstellungsversuche}
\subsection{Ermittlung der Architekturanforderungen}
\subsection{Erstellung der Architektur}
\subsection{Zusammenfassung}
\chapter{Anforderungen}
Die Anforderungen des/der KundIn sind die Basis des zu entwickelnden Systems und beschreiben nicht nur dessen Funktionalität, sondern auch dessen Qualitäten. Sie werden vom Kunden meist in der Form von Usecases beschrieben \cite[S. 78]{reqanalysis} und sind für die Erstellung der Architektur notwendig \cite[S. 9]{softarch}. Ein besonderer Fokus bei der Erstellung der Architektur liegt auf der geforderten Qualität des Systems, da eben diese stark von Architekturentscheidungen beinflusst wird \cite[S. 109]{softarch}: \glqq Architektur ist für die Qualität eines Systems notwendig\grqq \cite[S. 59]{effektiv}.

Sowohl funktionale als auch qualitative Aspekte sind für den Erfolg des Systems essentiell \cite[S. 109]{softarch}. Dies lässt an folgendem Beispiel demonstrieren: Ein/Eine KundIn beauftragt eine Firma, einen Webshop zu erstellen, auf welchem ihre/seine Produkte verkauft werden sollen. Die Funktionalität des Webshops wird wie beschrieben implementiert, aber das Endprodukt ist so langsam, dass ein Großteil der KundInnen den Bestellvorgang abbricht. Der Hauptgrund der Erstellung des Systems, nämlich Produkte verkaufen, ist somit zwar möglich, aber für den Kunden unrentabel.


Da Was ist Qualität \cite[S. 399]{pract}

ISO 9126 \cite{ISO_SQ} und dessen Nachfolger ISO 25010 \cite{ISO_SQ2} bieten ein Modell, um Softwarequalität zu beschreiben. ISO 25010 erweitert die in ISO 9126 beschriebenen Hauptkategorien um Security und Compatibility und fügt eine Zusätzliche Kategorie, nämlich Software Quality in Use hinzu, welche die eine eigene Qualitätskategorie für BenutzerInnen definieren. Da es aber schwierig war, zitierbare Quellen zum neuen Standard zu finden - ISO 25010 hat \glqq in die Praxis wenig Einzug gehalten\grqq \ \cite[S. 60]{effektiv} - , die Neuerungen überschaubar und mehr eine Umorganisierung als Revolution darstellen, wird auf die Nutzung von ISO 25010 verzichtet und das Qualitätsmodell des Vorgängers, ISO 9216, verwendet.



Behandeln der ISO 9126 Anforderungen, versuchen zu erklären was man alles beachten muss und was überhaupt ab anfang messbar ist (z.B. performance nicht messbar)
Ein wichtiges Ziel der Softwarearchitektur ist die Sicherstellung der Qualität des Systems \cite[S. 19]{effektiv}:

*

\section{Funktionale Anforderungen}
beschreibt die funktion die notwendig ist um das system für den kunden wertvoll zu machen\cite[S. 79]{reqanalysis}


Wichtig: Security ist eine funktionale Anforderung, ISO 25000 + IREB

\section{Nicht funktionale Anforderungen}
Einschränkungen, beschreiben nicht was sondern wie\cite[S. 80]{reqanalysis}

\subsection{Reliability}
\subsection{Usability}
\subsection{Efficiency}
\subsection{Maintainability}
\subsection{Portability}
\chapter{Softwarearchitektur}
Softwarearchitektur beschäftigt sich mit dem Erstellen des Fundamentes eines Softwareprojektes \cite[S. 9]{softarch}. Der Fokus liegt daher nicht auf Implementationsdetails sondern auf Schnittstellen, welche das System und die Zusammenarbeit der Teilkomponenten beschreiben \cite[S. 8]{basiswissen}. Ziel der Softwarearchitektur ist es, durch Abstraktion der Details einen Überblick über die zu implementierenden Komponenten zu gewinnnen, welcher es ermöglicht, frühe Entscheidungen über die Probleme und Angemessenheit des Systems zu erlangen und somit Risiken, Fehler- und Änderungskosten zu vermeiden oder zu verringern \cite[S. 10]{softarch}.

\section{Architektursichten}
Der Detailgrad der Schnittstellen hängt wesentlich von der Größe des Projektes ab \cite[S. 7-12]{basiswissen}. Um die Komplexität und Abstraktion zu beherrschen, können Architektursichten verwendet werden, welche einen Überblick über Teilbereiche des Systems ermöglichen \cite{ISO_ARCH}. Ein Beispiel dafür ist Kruchtens 4+1 Sichtenmodell \cite{kruch}, welches folgende Sichten beschreibt:

\begin{itemize}
  \item Logical View: Logische Sicht auf das System, welches die Klassen und deren Beziehungen untereinander darstellt
  \item Process View: Beschäftigt sich mit nicht funktionalen Anforderungen wie Performance und Fehlertolleranz und deren Implementierung
  \item Development View: Beschreibt die Aufteilung des Systems in Module und Bibliotheken
  \item Physical View: Regelt und beschreibt die Verteilung der Komponenten auf physische Systeme
  \item Scenarios: Usecases, welche die Nutzung des Systems beschreiben
\end{itemize}

Kruchten verwendet eigene Diagramme für jede Sicht, die an UML angelehnt aber zum Teil verschieden sind, etablierte sich UML doch erst zwei Jahre später nach der Veröffentlichung seines Modelles \cite{glasklar}[S. 5].

Es gibt noch viele andere Sichtenmodelle wie das Zachman Framework \cite[S. 282]{zachman}, welches eine Aufteilung in Kontext-, Geschäfts, System, Technologie, Integrations und Laufzeitsicht vornimmt. Die Gemeinsamkeit aller Sichtenmodelle ist jedoch der Versuch, durch unterschiedliche Sichtweisen einen besseren Überblick über das Zielsystem zu erlangen.



\section{Architekturprozess Lifecycle}
Zeitlich ordnet sich die Architekturphase zwischen dem Ermitteln der Anforderungen und der tatsächlichen Implementation ein. Da die Erstellung von Software heutzutage mehr ein agiler und inkrementeller Prozess ist, wird auch die Erstellung und die Mitwirkung der Softwarearchitektur in diese sich wiederholenden Prozesse mit einbezogen. \cite[S. 7]{basiswissen}

Der Architekturzyklus lässt sich mit dem in Abbildung \ref{fig:cycle} beschriebenen Aktivitätsdiagramm darstellen und unterteilt sich in folgende Unterbereiche \cite[Umschlag]{softarch}:

\begin{itemize}
  \item \glqq Erstellen der Systemvision\grqq
  \item \glqq Verstehen der Anforderungen\grqq
  \item \glqq Entwerfen der Architektur\grqq
  \item \glqq Umsetzen der Architektur\grqq
  \item \glqq Kommunizieren der Architektur\grqq
\end{itemize}


\begin{figure}[H]
    \centering
    \includegraphics[scale=0.8]{img/archcycle.png}
    \caption{Der Architekturprozess modelliert als Aktivitätsdiagramm \cite[Umschlag]{softarch}}
    \label{fig:cycle}
\end{figure}

Der Fokus der Arbeit liegt wegen der fehlenden Implementierung hauptsächlich auf den ersten drei Teilen, der Erstellung der Systemvision, Verstehen der Anforderungen und dem Entwerfen der Architektur.

\section{Was macht eine gute Softwarearchitektur aus}
Die Frage, was eine gute Softwarearchitektur ausmacht, wird oft sehr breit und ungenau beantwortet: Eine gute Architektur erfülle die \glqq Verhaltens-, Qualitäts- und Lebenszyklusanforderungen\grqq \ \cite[S. 12]{basiswissen} eine Systems, ermögliche es Risiken frühzeitig zu erkennen und die inhärente Komplexität eines Systems und dessen Quellcodes zu beherrschen \cite[S. 7-8]{softarch}.

Alle diese Kriterien lassen sich jedoch schwierig oder gar nicht messen. Manche Werte sind zwar messbar, aber lassen keine eindeutigen Entscheidungen ableiten: zB. können Kopplung und Kohäsion eines Systemes gemssen werden, jedoch kann nicht eindeutig festgelegt werden, ab welchem Wert die Kopplung zu hoch oder die Kohäsion zu niedrig ist.

Dieter Masek kommt zu folgendem Schluss: \glqq Die Frage ob eine gegebene Architektur gut oder schlecht ist, lässt sich nicht direkt beantworten. Gut oder schlecht sind absolute Bewertungen, die für Architekturen sowieso nicht möglich sind. Eine Architektur lässt sich nur in einem festgelegten Kontext beurteilen d.h.: Wie gut löst eine Architektur ein vorgegebenes Problem? Selbst diese eingeschränkte Frage lässt sich nicht mit gut oder schlecht beantworten! \grqq \ \cite[S. 19]{review}

Trotz all dieser Probleme lassen sich dennoch Werte messen und überprüfen \cite[S. 19]{review}, welche in Verbindung mit unzureichenden Architektureintscheidungen gebracht werden können. Dies erlaubt folgenden Schluss: Die Ausgangsfrage, was eine gute Softwarearchitektur ausmacht, führt zu keiner Erkenntnis und ist damit mehr oder minder nutzlos: Es kommt nicht darauf an, ob eine Architektur gut ist, sondern ob eine Architektur die an sie gestellten Anforderungen erfüllt.

Viele dieser Anforderungen stehen im Gegensatz zueinander: zB. erfordert eine hohe Performance eine niedrigere Abstraktionsebene, was jedoch die Wartung des Codes erschwert. Die Findung einer angemessene Softwarearchitektur beschäftigt sich somit mehr mit der Abwägung von Vor- und Nachteilen, welche aus den jeweiligen Entwurfsmustern und Technologien ableitbar sind.

Um diese Entscheidungen abwägen zu können, ist wichtig, die expliziten Anforderungen an das System zu kennen \cite[S. 19]{review}. Eine möglichst vollständige Anforderungsanalyse ist somit die Ausgangsbasis für eine angemessene Softwarearchitektur. Dies kann jedoch vor allem am Anfang der Anforderungsanalyse schwierig sein, da die KundInnen noch keine genaue Vorstellung vom zu entwickelnden System haben \cite[S. 80]{reqman}.

Eine Möglichkeit zur Überprüfung von bestehenden und noch nicht beachteteten Anforderungen der Architektur stellt die Durchführung eines szenariobasierten Architekturreviews wie ATAM oder CBAM dar.


\section{Architekturbewertungsmethoden}
Kurze Einführung in Architekturbewertungsmethoden

\subsection{ATAM}
\subsection{CBAM}
\chapter{Modellierung}
Kurze Einführung in die Modellierung. Warum ist sie wichtig, was wird verwendet. Die wichtigsten Grundsachen erklären und erklären für was welcher Diagrammtyp verwendet wird

\cite[S. 13]{reqanalysis}
\cite[S. 139]{effektiv}

\section{Warum ist Modellierung wichtig}
\section{Was ist UML}

\section{Diagrammtypen}
\subsection{Kontextdiagramm}
\subsection{Komponentendiagram}
\subsection{Klassendiagramm}
\subsection{Aktivitätsdiagramm}
\subsection{Usecasediagramm}
\chapter{Prozesserstellung}

sagen dass man eine art kochrezept will

\section{Vorhandene Daten}
Ausgangsparameter test
\subsection{Wer (Akteure}
\subsection{Was (Daten)}
\subsection{Warum (Anforderungen/Usecases)}
\subsection{Womit (Komponenten)}

\section{Prozesserstellungsversuche}
Ausgehend von den vorhandenen Daten wurden mehrere Prozesse definiert, welche bis auf den Letzten entweder zu grobe Ergebnisse lieferten, oder nicht nachvollziehbar waren.

Ausgangsbasis war eine Systemvision mit folgenden Anforderungen:

\begin{itemize}
  \item Es soll eine Webseite entstehen, welche die Prüfungstermine auflistet und Personen erlaubt, sich für diese Prüfungen anzumelden
  \item Die Übermittlung der Prüfungsdaten soll über einen eigenen VPN Server geschehen
  \item Die Prüfungsdaten werden firmenintern verwaltet und nach der Auswertung soll der Scheme Owner benachrichtigt werden
\end{itemize}

\begin{figure}[!htbp]
    \centering
    \includegraphics[scale=0.6]{uml/vision.png}
    \caption{Systemvision der Komponenten}
\end{figure}

Aufbauend darauf wurde dann versucht einen Prozess zu finden, der diese Grundideen berücksichtigt.

\subsection{Vom Usecase zur Komponente}
Der erste Versuch zur Erstellung des Architekturprozesses orientierte sich am Prinzip: teile und herrsche. Der Prozess hing stark von den nicht funktionalen Anforderungen ab und sollte auf folgender Weise funktionieren:

\begin{itemize}
  \item Für jeden Usecase wird ein komplettes Komponentendiagramm des Systems erstellt
  \item Die Komponenten jedes Teilsystems werden anhand ihrer nicht funktionalen Qualitäten aus einem Pool von Komponentenarchitekturen gewählt
  \item Schlussendlich werden alle Teilsysteme mit einander vereinigt, soweit es die nicht funktionalen Attribute erlauben
\end{itemize}

Dieser Prozess scheiterte nicht nur am enormen Modellierungsaufwand, sondern auch am Auswahlprozess der Komponenten: Je nachdem, welche Komponentenarchitekturen vorhanden waren und wie diese bewertet wurden, entstanden unterschiedliche Architekturen. Zudem schien es zu viele Komponentenarchitekturen zu geben, da die einzelnen Komponenten beliebig miteinander kombinierbar waren.

Die Qualität der Architektur hätte folglich von der Vollständigkeit dieser scheinbar unendlich großen Menge an Komponentenarchitekturen abgehangen. Aus diesem Grund schien er ungeeignet und wurde verworfen.

\subsection{Von einer geschätzten Architektur auf eine verfeinerte Architektur durch nicht funktionale Anforderungen}
Zweiter Versuch: Nach Erfahrungswerten so viel wie möglich erstellen, danach überprüfen auf nicht funktionale Anforderungen und notwenige Änderungen einbringen -> kein Kochrezept, verlässt sich zu viel auf Erfahrungswerte
\subsection{Von den Daten zur Architektur}
Dritter Versuch: Aufspalten der Architektur in Datenbereiche. Ging schon gut aber Ergebnis zu grob
\subsection{Von den Daten und den Akteuren zur Architektur}
Vierter: Einbeziehen der Akteure

\chapter{Prozess Anforderungen}
Da die Software Architektur auf den Anforderungen basiert und viel wichtiger \glqq den gestalterischen Spielraum des Architekten\grqq \cite[S. 103]{softarch} begrenzt, kann davon abgeleitet werden, dass sowohl die Qualität der Architektur als auch die Akzeptanz des Systems wesentlich von den bereits im Vorfeld ermittelten Parametern abhängt. Das bedeutet wiederum, dass für die Klärung der Ausgangsfrage - Wie kommt man von Anforderungen auf eine gute Architektur - auch der Anforderungsprozess eine wichtige Rolle spielt.

Da der Anforderungsprozess ein an sich eigenes, sehr großes Themengebiet dar stellt, wird hier jedoch nur auf die Ausgangsartefakte eingegangen, welche später im Architekturprozess referenziert werden.

Die Entwicklung der meisten Software Projekte ist oft ein dynamischer und agiler Prozess. Während des Entwicklungsprozesses werden oft Änderungen eingebracht und neue Anforderungen aufgestellt  \cite[S. 6-7]{effektiv}. Auch führt die Komplexität des Systems und das Wissen des/der KundIn dazu, dass nach der Erhebung der initialen Anforderungen nicht alle Parameter vollständig bekannt sind. Dies wiederum führt mehr oder weniger dazu, dass der Anforderungsprozess während der Entwicklung der Architektur nicht als abgeschlossen gesehen werden kann und der/die KundIn verfügbar sein muss. Dieses scheinbare Problem erlaubt es jedoch, den Aufwand der Aufnahme bestimmter Anforderungen zu minimieren, da bereits viele Kombinationen ausgeschlossen werden können.

\section{Ermittlung der Usecases}
Die Usecases werden zusammen mit dem/der Kundin ermittelt. Daraus wird schlussendlich ein Usecase Diagramm erstellt, welches alle Akteure und Nebensysteme beinhaltet. Dies ist wichtig für das Kontextdiagramm, welches auch im Anforderungsprozess erstellt wird und die Ausgangsbasis für die Architektur dar stellt.

\begin{figure}[!htbp]
    \centering
    \includegraphics[scale=0.4]{uml/usecase.png}
    \caption{Das mit dem Kunden ermittelte Usecase Diagramm}
\end{figure}

\subsection{Erweiterte Dokumenation der Usecases}
Parallel zur Erstellung des Usecase Diagramms werden zusätzliche Parameter und Beschreibungen für jeden Usecase aufgenommen, welche im eigentlichen Diagramm keine Erwähnung finden.
Dafür wird ein Anforderungstemplate, auch Usecase Beschreibung genannt, verwendet \cite[S. 214]{reqman}, welches aufbauend auf einer Grundversion \cite[Abbildung 8.14, S. 215]{reqman} für jeden Usecase folgende Angaben aufnimmt:

\begin{itemize}
  \item Id: eine eindeutige Bezeichnung, welche dafür verwendet wird, um den Usecase zu referenzieren
  \item Actor: eine Auflistung aller Teilnehmer des Usecases
  \item Description: eine kurze Beschreibung des Usecases
  \item Preconditions: eine Auflistung von Vorbedinungen für den Usecase
  \item Postconditions: eine Auflistung von Nachbedingungen für den Usecase
  \item Normal Course of Events: eine Beschreibung des Standardablaufes
  \item Alternative Courses: Auflistung von Erweiterungen oder zusätzlichen Möglichkeiten
  \item Exceptions: Beschreibung von diversen Ausnahmefälle
  \item Assumptions: Annahmen, unter welcher der Usecase beschrieben wird
  \item Priority: eine Gewichtung, wie wichtig der Usecase ist
  \item Notes: sonstige Anmerkungen
\end{itemize}

Sind die Abläufe komplexer, können Aktivitäts Diagramme verwendet werden, um komplexere Abläufe verständlicher dar zu stellen \cite[S. 215]{reqman}:

\begin{figure}[H]
    \centering
    \includegraphics[scale=0.4]{uml/takeexamreq.png}
    \caption{Der Ablauf des Take Exam Usecases im Detail}
\end{figure}

\subsection{Einbeziehen von Architekturreviewparametern}
Um die Einhaltung der Qualitätsparameter zu garantieren, können Architekturreviews durchgeführt werden \cite[S. 20]{review}. Es existiert zwar \glqq keine singuläre, allgemein akzeptierte Metrik um eine Architektur zu beurteilen\grqq \cite[S. 19]{review}, jedoch liefern sie grobe Einschätzungen über die Angemessenheit des Systems \cite[S. 20]{review}. Folgende Architekturreviews wurden dafür ausgewählt:

\begin{itemize}
  \item ATAM: betrachtet Wachstums- und explorative Szenarien um die Architektur zu beurteilen \cite[S. 61]{review}
  \item CBAM: basiert auf ATAM, beachtet jedoch vor allem den Nutzen und die Risiken und Kosten der Architketur, um die Architekturentscheidungen besser abwähgen zu können. Hauptfaktor ist der ROI. \cite[S. 67]{review}
\end{itemize}

Für diese Reviews können bereits früh ein Großteil der benötigten Parameter ermittelt bzw. zumindest grob abgeschätzt werden. Dies ist wichtig, weil nach der 10er Regel der Fehlerkosten früh erkannte Fehler und Probleme weniger Kosten nach sich ziehen als später Erkannte \cite[S. 154]{fehler}.

Deswegen wird das Anforderungstemplate um folgende Parameter erweitert:

\begin{itemize}
  \item Earned Value per month: Wieviel Umsatz der Usecase in einer bestimmten Zeit generiert
  \item Expected Usage: Anzahl der erwarteten Nutzer des Systems pro Zeiteinheit
  \item Growth Scenarios: Anzahl der erwarteten Nutzer des Systems pro Zeiteinheit bei einer höheren Nutzeranzahl
  \item Change Scenarios: mögliche Änderungsszenarien und Erweiterungen
\end{itemize}

\subsection{Einbeziehen von nicht funktionalen Qualitätsattributen}
Nicht funktionale Qualitätsattribute beschreiben die nicht funktionalen Anforderungen an das System. Da Qualität oft schwammig formuliert ist, ist es wichtig, diese Attribute messbar zu machen \cite[S. 9]{effektiv}.

Deswegen werden für jeden Usecase und dessen architekturrelevanten, nicht funktionalen Anforderungen messbare Parameter definiert. Diese Parameter helfen schon im Vorfeld dabei die Architektur zu überprüfen.

Für das Beispielprojekt wurden folgende Parameter definiert, mit welchen das Anforderungstemplate erweitert wurde:

\begin{itemize}
  \item Response Time in Seconds: Wie schnell die Antwort des Systems auf eine Anfrage reagieren muss
\end{itemize}

\section{Rahmenbedingungen}
Zusätzlich zu funktionalen und nicht funktionalen Anforderungen werden auch die Rahmenbedingungen ermittelt, unter welchem das System erstellt werden soll. Diese Anforderungen beinhalten meist den organisatorischen und zeitlichen Ablauf des Projektes und können auch gewisse Technologien vorschreiben, zB. wenn das System in ein bereits bestehendes System integriert werden soll. \cite[S. 9]{review}\cite[S. 110]{softarch}

Die Rahmenbedingungen des Beispielprojektes lassen sich zum Großteil aus dem ISO Standard für Zertifizierungsstellen ermitteln \cite{ISO_CERT} und geben Einsicht in die Vertraulichkeit der Daten und eröffnen weitere Usecases. Sofern möglich werden diese Parameter in das Usecase Diagramm und das Klassen Diagramm der zu verwendeten Daten mit einbezogen. Auf zeitliche und technologische Rahmenbedingungen wurde im Beispielprojekt nicht eingegangen.

\section{Ermittlung der Daten}
Die verwendeten Daten werden ermittelt und mit Hilfe eines Klassen Diagrammes modelliert. Dies ist nicht nur wichtig und nützlich, weil die Beispielanwendung in diesem Falle eine stark datenzentrierte Anwendung ist \cite[S. 105]{effektiv}, sondern wird später auch einen wesentlichen Beitrag zur Aufteilung des Systems in Komponenten leisten. Im Falle des Beispielprojekts wurden folgende Daten ermittelt und modelliert:

\begin{figure}[H]
    \centering
    \includegraphics[scale=0.5]{uml/class.png}
    \caption{Das ermittelte Klassendiagramm des Beispielprojektes}
\end{figure}

\section{Ermittlung der Zonen}
Nach der Ermittlung der Daten wird auf Basis der Rahmenbedingungen und Sicherheitsstruktur zusammen mit dem/der Kunden/Kundin ermittelt, welche Zonen für das System benötigt werden. Zonen stellen eigene abgeschlossene Bereiche dar, in welchen das System operiert oder von welchem auf das System zugegriffen werden kann. Grundsätzlich existiert mindestens eine Zone. In diesem Fall beherbergt diese Zone das interne System des Unternehmens.

Soll von anderen Systemen, zB. vom Internet auf Funktionalitäten des Systems zugegriffen werden können, wird eine weitere Zone benötigt.

Aus den Usecases des Beispielprojektes lässt sich ermittelt das in diesem Falle mindestens zwei Zonen benötigt werden:

\begin{itemize}
  \item Public: Usecases die vom Internet auf das System zugreifen, zB. wenn sich ein Anwärter (Applicant) für eine Prüfung registriert
  \item Internal: Usecases, welche nicht vom Internet aus zugänglich sind
\end{itemize}

Die Zonen werden nun mit einer Vertrautheitsebene von 1 bis n versehen, um eine Sicherheitshirarchie der Zonen zu generieren. Es kann vorkommen, dass mehrere Zonen mit der selben Vertrautheitsebene versehen werden, diese dürfen sich jedoch nicht überlappen. Zum Beispiel könnte eine weitere Tocherniederlassung der Firma ein eigenes System erstellen, auf welches man auch vom Internet aus zugreifen kann. Das Internet würde dann mit der Vertrautheitsebene 1 versehen und beide Systeme mit der Ebene 2. Überlappen sich diese Systeme, muss die überlappende Zone in ein eigenes System ausgegliedert werden.

Die ermittelten Daten werden nun den Zonen, in welchen sie essentiell für den Betrieb des Systemes sind, zugeteilt. Auch die Akteure werden in Zonen aufgeteilt, in welchen sie operieren. Scheinen Daten oder Akteure in mehreren Zonen auf, werden sie der Zone mit der höchsten Sicherheitsebene zugeteilt. Die niedrigeren Bereiche können dann auf diese Bereiche anhand von festgelegten APIs zugreifen.

Im Falle des Beispielprojektes werden alle aufgenommen Daten in der Internal Zone verwaltet, da alle Daten als businesskritisch für das Internal System angesehen werden, welches wiederum die höchste Sicherheitsebene besitzt. Würde das Beispielprojekt zB. um ein Forum oder einen Blog erweitert, würden diese der Public Zone zugeteilt werden; AkteurInnen des Internal Systems können zwar auch auf diese Forum zugreifen oder Blogeinträge schreiben, jedoch sind diese Aktivitäten und Daten nicht kritisch für den normalen Betrieb des Internal Systems. Würde das Forum oder der Blog gehackt werden, resultiert dies zwar in einem Schaden für das Unternehmen, das Internal System könnte aber auch weiterhin normal operiert werden.

Nach der generellen Aufteilung der Daten wird eine Analyse der Rahmenbedingungen durch geführt, um zu erfahren, ob bestimmte Daten aufgrund von Richtlinien besonders geschützt, und somit in eine eigene Zone mit einer höheren Vertrautheitsebene ausgelagert werden müssen. Das Beispielprojekt verlangt die vertrauliche Verarbeitung von Prüfungsergebnissen und erwähnt auch explizit das Personal der Prüfungsstelle \cite[7.3]{ISO_CERT}. Weil die Internal Zone, in welchem das Personal operiert, die Zone mit der höchsten Vertrautheitsebene darstellt, muss eine weitere Zone mit einer höheren Vertrautheitsebene erstellt werden. Diese Zone wird in diesem Falle mit der Vertrautheitsebene 3 versehen und unter der Beschreibung Confidential geführt.

Um diese Zonen besser zu visualisieren zu können, wird das UML Metamodel mit Hilfe eines Profiles angepasst. Jede Zonen erhält einen gleich lautetenden Stereotypen \cite[S. 518]{glasklar}:

\begin{figure}[H]
    \centering
    \includegraphics[scale=0.5]{uml/datastereotypes.png}
    \caption{Das Metamodell wird mit einem Profil um drei Stereotypen erweitert, welche die Zonen der Applikation darstellen}
\end{figure}

Zusätzlich werden diese Zonen auch mit Vertrautheitsebenen verbunden. Diese Ebenen werden im Profil mit einer Notiz versehen, welche die Ebene angibt.

\begin{figure}[H]
    \centering
    \includegraphics[scale=0.5]{uml/datastereotypeslevel.png}
    \caption{Die Zonen werden mit Vertrautheitsebenen versehen}
\end{figure}

Sind die Stereotypen erstellt und mit Vertrautheitsebenen versehen, kann nun damit begonnen werden, die AkteurInnen des Usecase Diagrammes und die Daten des Klassen Diagrammes mit diesen Stereotypen zu versehen.

\begin{figure}[H]
    \centering
    \includegraphics[scale=0.5]{uml/classstereotyped.png}
    \caption{Das Klassendiagramm wird mit Stereotypen der Vertraulichkeit erweitert}
\end{figure}

\begin{figure}[H]
    \centering
    \includegraphics[scale=0.4]{uml/stereotypedusecase.png}
    \caption{Die AkteurInnen des Usecase Diagrammes wird mit Stereotypen der Vertraulichkeit erweitert}
\end{figure}

\section{Ermittlung der Beziehungen zwischen Akteuren/Partnersystemen und Daten}
Auf Basis des Usecase Diagrammes können die Akteure und deren Partnersysteme mit Hilfe eines Kontext Diagrammes visualisiert werden. Im Gegensatz zum Usecase Diagramm geht das Kontext Diagramm auf die zwischen den Systemen und AkteurInnen fließenden Daten ein und stellt so die Verbindung der Daten und Nutzer auf.

\begin{figure}[H]
    \centering
    \includegraphics[scale=0.5]{uml/context.png}
    \caption{Das Kontext Diagramm zeigt das System, die Akteure und die Nachbarsysteme}
\end{figure}
\chapter{Prozess Architekturplanung}
Aufbauend auf den im Anforderungsprozess ermittelten Attribute, kann nun mit der Architekturplanung begonnen werden.

\section{Erstellen der Minimalen Architektur}
Das Kontextdiagramm, welches im Anforderungsprozess erstellt worden ist, zeigt das System mit allen AkteurInnen und Nachbarsystemen. Aufbauend darauf kann nun die minimale Architektur erstellt werden, welche sich aus dem System und den Nachbarsystemen ableitet.

\begin{figure}[H]
    \centering
    \includegraphics[scale=0.5]{uml/context.png}
    \caption{Das Kontextdiagramm liefert die Ausgangsbasis für die Architektur}
\end{figure}

Zuerst werden alle Datenflussnotizen entfernt. Danach werden alle Komponenten entfernt, welche kein eigenes System darstellen. In diesem Falle werden folgende Komponenten entfernt:

\begin{itemize}
  \item Applicant
  \item Certification Body
  \item Invigilator
\end{itemize}

Dies führt zu folgender Minimalarchitektur:

\begin{figure}[H]
    \centering
    \includegraphics[scale=0.7]{uml/minimalarch.png}
    \caption{Minimale Architektur}
\end{figure}

Für die Nachbarsysteme selbst wird keine Architektur erstellt, jedoch beeinflussen sie die Schnittstellen des Systems und sind deswegen wichtig für den weiteren Prozess. Sie werden in die Architektur einbezogen.

\section{Erstellen der Datenminimalarchitektur}
Auf Basis der im Anforderungsprozess ermittelten Zonen wird das System der vorher erstellte Minimalarchitektur in ebenso viele Teilsysteme unterteilt. Die Aktivitätsdiagramme werden an die neue Architektur angepasst: Für jedes Untersystem wird in den Diagrammen eine eigene Swimlane erstellt. Die involvierten AkteurInnen sind wenn möglich als eigene Swimlane modelliert, spielen in dieser Phase aber noch keine wichtige Rolle zur Gliederung des Systems.

\begin{figure}[H]
    \centering
    \includegraphics[scale=0.5]{uml/takeexamactivity1.png}
    \caption{Die Antworten werden nach der Prüfung an den Certification Body übermittelt. Der Request wird dann durch zwei Gateways zum finalen System geleitet.}
\end{figure}

Wechselt der Kontrollfluss eine Swimlane eines Systems, heißt dies, dass eine Verbindung zwischen den beiden sonst abgeschotteten Systemen benötigt wird. Dieses Verbindung wird als eigene Komponente modelliert und wird als Gateway bezeichnet. Die Aufgabe dieses Gateways ist es, folgende Attribute der Anfrage zu überprüfen und die Anfrage gegebenenfalls zu verwerfen oder weiterzuleiten:

\begin{itemize}
  \item Von welchem System kommt die Anfrage?
  \item Welches System ist das Ziel der Anfrage?
  \item Welche Schnittstelle dieses Systems ist das Ziel der Anfrage?
  \item Gibt es eine Regel die diese Anfrage explizit erlaubt?
\end{itemize}

Diese Komponenten fungieren damit als eine Art Application Firewall.

Die anfangs beschriebenen Nachbarsysteme werden nach ihren Anforderungen, welche aus den Aktivitätsdiagrammen ablesbar sind, entweder an das System in ihrer Zone angeschlossen, oder direkt mit dem Gateway verbunden.

Das Beispielprojekt bezieht Zahlungsdaten direkt von einem Payment System und die Prüfungsfragen werden direkt an das Scheme Owner System gesandt. Für diese beiden Usecases ist somit kein Zwischensystem notwendig und die Anfragen können vom Internal System direkt an das Ziel System geschickt werden. Dem entgegen gesetzt ist die Übermittlung der Prüfungsfragen des Scheme Owner Systems: Das System braucht eine durch Internet erreichbare Schnittstelle um die Daten zu übermitteln, da das Ausgangssystem nicht bekannt oder variabel ist und muss deswegen zuerst über das Public System zum Ziel.

\begin{figure}[H]
    \centering
    \includegraphics[scale=0.7]{uml/dataarch.png}
    \caption{Aufteilung der Komponenten in Datenbereiche}
\end{figure}

Wichtig ist hier, dass keine Gateways unterschiedlicher Vertrautheitsebenen übersprungen werden. Zeigt ein Aktivitätsdiagramm zB. einen Zugriff von Ebene 1 auf Ebene 3 muss dieser Zugriff sowohl durch den Gateway der Ebene 2 geleitet werden, als auch durch den Gateway der Ebene 3. Dies verhindert, dass besonders schützenswerte Systeme direkt an Systeme mit einer weitaus niedrigeren Vertrautheitsebene angeschlossen werden und so dessen Gateway zum Single Point of Failure wird. Dies gilt in beide Richtungen.

Da bei der Erstellung des Systems alle Schnittstellen und Systeme bekannt sind, können diese Regeln fest im Gateway verankert werden. Weil diese Gateways unabhängig voneinander agieren, können sie durch das Hinzufügen eines Load Balancers beliebig vervielfacht werden, was sowohl die Ausfallsicherheit als auch die Skalierbarkeit erhöht. Das ist wichtig, weil sie als einzige Verbindung zwischen den Systemen zu einer Art Flaschenhals werden.

\section{Einbinden der AkteurInnen}
Nachdem die Datenminimalarchitektur steht, können nun die AkteurInnen des Systems in die Aufgliederung des Systems mit einbezogen werden. Hierfür müssen nun die Objektflüsse und die AkteurInnen des Systems für jeden Usecase betrachtet werden, welche aus den vorher bereits erstellten Aktivitäts und Kontextdiagramm ersichtlich sind.

Zuerst wird das erste Untersystem, in diesem Falle das Public System, betrachtet. Alle Objektflüsse durch das System und die AkteurInnen, welche mit ihren Swimlanes angrenzen, sind in die Aktivitäten des Systems involviert. Jede Involvierung eines/einer Akteurs/Akteurin in ein System erfordert einen Zugang zu diesem System.

Jeder dieser Akteure muss mit den minimal möglichen Rechten für dieses System ausgestattet werden, um seine Aufgaben zu erfüllen. Dies vermeidet nicht nur Fehler sondern reduziert auch den Schaden, welcher ein potentieller Angriff dieses Akteurs/dieser Akteurin anrichten kann \cite[1. A]{leastpriv}.

Da ein System komplex ist \cite[S. 7]{softarch}, und diese Sicherheitsattribute nach Änderungen am System immer wieder überprüft werden müssen, stellt jeder zusätzliche Zugriff eines/einer Akteurs/Akteurin nicht nur ein Sicherheitsrisiko dar, sondern erhöht auch den Test- und damit den Wartungsaufwand. Idealerweise wird daher jedem/jeder AkteurIn ein eigenes, für sich abgekapseltes System zur Verfügung gestellt, was jedoch meist aufgrund Kosten der zusätzlichen Systeme keine Option dar stellt.

Um zu ermitteln, welche Systeme eine eigene Komponente benötigen, wird nun entweder anhand einer Tabelle oder zusammen mit dem/der KundIn pro Usecase und deren Komponenten ermittelt, ob der Schaden eines unerlaubten Zugriffs der Daten den eines Systems überschreitet. Die Schadens- und Systemkosten müssen zuerst von dem/der KundIn und dem/der ArchitektIn geschätzt werden.

Im Falle des Beispielprojektes wurde auf Basis des folgenden stark vereinfachten Aktivitätsdiagramms in Abbildung \ref{fig:actorarch} ermittelt, dass die möglichen Schadenskosten im Falle, dass der Anwärter (Applicant) Zugriff auf die Prüfungsantworten (Answer) bekommt, die eines eigenen Systems überschreiten. Das gleiche Problem trifft auch auf den Scheme Owner zu: die Schadenskosten im Falle einer Manipulation oder eines lesenden Zugriffes des Anwärters (Applicant) auf die Fragen überschreitet auch hier die Kosten eines eigenen Systems. Deswegen werden zwei zusätzliche Systeme erstellt und aus dem Public System ausgegliedert.

Für den Fall, dass bei der Aufspaltung zu viele Systeme entstanden sind, werden nun in einem weiteren Schritt diverse Kombinationen von Teilsystemen betrachtet und versucht zusammen zu legen, solange deren Schadenskosten nicht die Systemkosten überschreiten.

Beim Beispielprojekt zeigt sich, dass es keine erlaubte Kombination gibt, da die anderen AkteurInnen nicht auf diese Daten zugreifen dürfen. Es bleibt somit bei den ermittelten zwei Zusatzsystemen.

\begin{figure}[H]
    \centering
    \includegraphics[scale=0.6]{uml/actorarch.png}
    \caption{Vereinfachte Gegenüberstellung von Aktivitätsdiagramme für das Public System}
    \label{fig:actorarch}
\end{figure}

Diese Analyse wird für alle verbleibenden Systeme durchgeführt, bis alle Systeme aufgespalten sind.

Im Falle des Beispielprojektes führt dies schlussendlich zu folgender Systemaufspaltung:

\begin{figure}[H]
    \centering
    \includegraphics[scale=0.6]{uml/vision4.png}
    \caption{Architektur nach der Aufspaltung }
\end{figure}


\section{Modellieren der Komponenten Interfaces (Klassen Diagramm)}
Aufzeigen dass zb das interne System user anlegen können muss mit methoden im klassendiagramm

\section{Analyse der nicht funktionalen Attribute}
Auf Basis von dokumentierten Szenarien können nun nicht funktionale Attribute gemessen werden und Hinweise kritische/wichtige Komponenten gegeben werden. Kostengegenüberstellung können auch eigene Systeme rechtfertigen/entfernen

\subsection{Reliability}
Single Point of Failure Analyse (Matrix Komponente x Usecase), Erklären wie man auf Matrix kommt (Aktivitätsdiagramm), Ausfallskosten (inkl. Wachstumsszenarien)

Einfache Methode zur schätzung der Ausfallkosten, aufzeigen wie durch Reduzieren der Ausfallswahrscheinlichkeit Kosten sinken aber auch Investitionskosten verursachen. Wachstumsszenarien auch einbeziehen in die Rechnung
\subsection{Usability}
In diesem Teil des Prozesses nicht wichtig, da noch keine Implementierung vorhanden.

\subsection{Efficiency}
Efficiency kann pro usecase gemessen werden, zb für antwortzeiten indem man zb die swimlanewechsel der Aktivitätsdiagramme zählt und mit einer konstanten multipliziert (geschätzte Netzwerkgeschwindigkeit). Ansonsten ist es durch die fehlende Implementation nicht möglich die Geschwindigkeit oder den Arbeitsspeicherverbrauch zu messen.

\subsection{Maintainability}
Auslesbar aus der Usecase Matrix, als Summe aller subsysteme,

\subsection{Portability}
In diesem Teil des Prozesses nicht wichtig, da noch keine Implementierung vorhanden.

\chapter{Zusammenfassung}
In der Arbeit wurde anhand eines Beispielprojektes ein UML-basierter, allgemein anwendbarer Architekturprozess erstellt.


\section{Vorteile des erstellten Architekturprozess}
Soll ein Architekturprozess für die Softwareentwicklung eingeführt werden, so muss dieser einen Vorteil gegenüber des Status Quo bieten. Welche Vorteile zeichnen den in der Arbeit beschriebene Architekturprozess jedoch aus?

Die Vorteile ergeben sich im Prinzip aus den treibenden Anforderungen an den Prozess und das Projekt: Der Prozess soll nachvollziehbar sein, sodass verschiedene Personen mit den gleichen Anforderungen eine identische oder zumindest stark identische Architektur erstellen, der Fokus liegt auf der frühen Planungs- und nicht der Implementationsphase, das Projekt selbst ist ein mittelgroßes, typisches Projekt und erfordert durch dessen Anforderungen eine hohe Datensicherheit.

Daraus ergeben sich folgende Vorteile:

\begin{itemize}
  \item Gute Verständlichkeit
  \item Fokus auf Kosteneffizienz
\end{itemize}

\subsection{Gute Verständlichkeit}
Der Prozess ist aufgrund der Verwendung von UML in einer Modelliersprache dokumentiert, welche nicht nur eine große Bekanntheit und Verwendung findet, sondern auch oft im Anforderungsprozess verwendet wird. Der Architekturprozess kann damit auf dem Anforderungsprozess aufbauen und Modelle weiterentwicklen anstatt diese in komplett neue Modellierungen zu überführen. Dies hilft nicht nur bei der Kommunikation mit dem/der KundIn sondern auch mit dem Anforderungsteam: Wenn sich zB. Anforderungen ändern, können diese an den gleichen Modellen geändert werden. Die erstellten Modelle dienen zugleich auch als Dokumentation des Projektes und helfen bei der Wartung.

Zusätzlich profitiert der Prozess von den generellen Eigenschaften einer visuellen Modellierungssprache: Sie bildet das komplexe auf eine einfachere Darstellung ab und verringert die Information auf die wirklich wichtigen Bereiche. Durch die visuelle Komponente ist das Modell schneller aufnehm- und verstehbar als purer Prosatext.

Durch den Fokus auf Preis-Leistungsverhältnis ist es auch einfacher, die Probleme, Entscheidungen und Kosten dem Kunden zu kommunizieren, welcher wegen der fehlenden Vertrautheit mit dem Thema Softwarearchitektur oft die Entscheidungen und Notwendig dieses Bereiches anzweifelt \cite[S. 8-9]{softarch}. Da er mit Kostenfragen in der Regel vertraut ist, ergibt sich hier sogar eine Chance, bessere Anforderungen zu erlangen.

\subsection{Fokus auf Kosteneffizienz}
Durch die Fokussierung des Prozesses auf die frühe Architekturplanungsphase, lassen sich hier die meisten Fehlerkosten einsparen: Nach der 10er Regel der Fehlerkosten steigen die Fehlerkosten mit dem Projektfortschritt exponentiell an. Durch die Einbeziehung von Parametern, auf welche  Architekturreviewszenarien aufbauen lassen sich auch hier im Vorfeld Kosten reduzieren: Dies betrifft nicht nur die Ermittlung der Parameter selbst, sondern auch die Möglichkeit, triviale Probleme, welche erst bei einem Architekturreview offensichtlich werden können, durch diese Anforderungen schon im Vorfeld identifizieren und beheben zu können.

Eine weitere kostensparende Eigenschaft ist, dass architekturrelevante Parameter schon in der Anforderungsphase ermittelt werden und somit die Anzahl der Kundenkontakte reduziert werden kann.

Softwarearchitekturentscheidungen sind oft ein Trade-Off: Durch konkurrierende Qualitätsanforderungen, zB. Performance und Wartbarkeit, ist es in den seltesten Fällen möglich, eine Architektur zu erstellen, welche in allen Qualitätsanforderungen brilliert. Aus diesem Grund ist es notwendig, eine bestimmte Vorgehensweise zur Erstellung, Bewertung und Entscheidung von Architekturen zu pflegen. Der Prozess bietet in dieser Hinsicht ein kostenbasiertes Entscheidungs- und Analysemodell an, mit welcher diese Entscheidungen gegenrechenbar werden. Dadurch, dass weitere Systeme nur dann erstellt werden, wenn sie einen Kostenvorteil bergen, entsteht eine kosteneffiziente und preislich angemessene Architektur.

Nicht nur tatsächliche Kosten, sondern auch Risikokosten werden im Architekturprozess mit einbezogen. Der Fokus des Prozesses liegt auf Angriffs-, Aufalls- und Änderungsszenarien, welche als Resultat des Umfeldes des Beispielprojektes gesehen werden können.

Ein weiterer Kostenvorteil ist die Möglichkeit, Anschaffungen und Architekturentscheidungen aufzuschieben, da sich der Prozess nicht auf eine physische Architektur fest legt. Die wichtigen Grundsteine, die Schnittstellen und Kommunikationswege, sind bereits geregelt, was zusätzlich eine Paralellisierung der Architekturerstellung erlaubt.


\section{Limitierungen, Probleme und Nachteile des erstellten Architekturprozesses}
Der Prozess ist zwar allgemein anwendbar, ist aber aufgrund der Herangehensweise für manche Gebiete schlechter geeignet. Die Probleme des sind vor allem in folgenden Bereiche auffindbar:

\begin{itemize}
  \item Ausgreift- und Erprobtheit
  \item Vollständigkeit
  \item Universelle Anwendbarkeit
\end{itemize}

\subsection{Ausgreift- und Erprobtheit}
Der Prozess wurde erst anhand eines einzigen Projektes erprobt und ging nicht über die Planungsphase der Architektur hinaus. Die Implementierungsphase und abschließende Überprüfung der implementierten Software wurde aus Zeit- und Umfangsgründen nicht durchgeführt. Dies führt nicht nur zu einer mangelnden Feedbackphase, in welcher Architekturprobleme während der Implementation offensichtlich werden hätten können, sondern auch zu einem zu starken Fokus auf die Domäne des Beispielprojekts.

Dies wird unter Anderem darin offensichtlich, dass der Prozess einen großen Wert auf Datensicherheit legt, welcher durch die dem Projekt zugrunde liegenden Anforderungen enstehen: Die Sicherheit und der richtige Umgang mit den Kundendaten ist eine bindende Anforderung zur Erfüllung des ISO Standards für Zertifizierungsstellen, ohne welchen das System nicht betrieben werden könnte. \cite{ISO_CERT}

Der Prozess ist somit noch unzureichend für den Realeinsatz getestet. Vor Allem der Einsatz in unterschiedlichen Applikationsdomänen muss noch stärker geprüft werden.

\subsection{Vollständigkeit}
Die Probleme der Vollständigkeit des Prozesses sind unter Anderem eine Konsequenz der geringen Erprobtheit. Da es sehr viele unterschiedliche Risiken gibt und für unterschiedliche Projektfelder unterschiedliche Qualitäten verlangt werden, sind im Prozess nur ein Teil der möglichen Bereiche bearbeitet worden.

Außerdem behandelt der Prozess nur die Planungsphase und nicht die Architekturphase und beschäftigt sich somit weder mit der Strukturierung des Codes, noch gibt er Auskunft über die tatsächlichen eingesetzten, physischen Komponenten.


\subsection{Universelle Anwendbarkeit}
Der Architekturprozess ist vor allem geeignet für Informationssysteme, deren Fokus auf folgenden Anforderungen liegt:

\begin{itemize}
  \item Daten unterschiedlicher Vertraulichkeit müssen verarbeitet werden
  \item Datensicherheit ist wichtig
  \item Nicht jede AkteurIn kann auf alle Daten zugreifen
  \item Preis-Leistungsverhältnis ist wichtig für die Erstellung der Software
\end{itemize}

Liegt der Fokus jedoch auf anderen Bereichen, kann das Resultat der Analysen eine sehr simple Architektur sein, welche im Implementationsprozess stark angepasst werden müssen.

Ein Beispiel hierfür ist das Erstellen einer Architektur für ein sehr rechenintensives Systems, zB. für die Berechnung von Primzahlen. Hier ist es wichtig, die Berechnungen gut zu paralellisieren; der Architekturplanungsprozess würde aufgrund der Anforderungen eine einzige Softwarekomponente zur Berechnung der Primzahlen ermitteln, was zwar richtig wäre, aber wenig Hinweise zur Strukturierung der tatsächlichen physischen Systeme bieten würde.

Eine zweites Beispiel wäre ein Embedded System, welches eine starke Limitierung des Netzwerk- und Speicherdurchsatzes besitzt: Durch die Aufspaltung in mehrere Systeme, welche dann durch ein Netzwerk verbunden werden, würde die Zeit, welche zur Übermittlung der Daten aufgewendet würde stark unter der ermittelten Architektur leiden. Auch eine Virtualisierung der Komponenten auf einem System würde wegen der Speicherlimittierungen nicht in Frage kommen. Die ermittelte Architektur müsste aufgrund der Hardware Limittierungen stark überarbeitet werden.

\section{Erkenntnisse}
\subsection{Ohne messbare Parameter kein Kochrezept möglich}
\subsection{Parameter nicht zu jeder Phase messbar}
\subsection{Priorisierung von nicht funktionalen Parametern schwer möglich}
\subsection{Generische Komponentenarchitektur durch viele und schnell ändernde Kombinationen schwer möglich}
\subsection{Auf Ehrfahrungswerte kann nicht vollkommen verzichtet werden}
\subsection{Funktionale Anforderungen beeinflussen Archtitektur mehr als gedacht}
Prozess erstellt die frühe Architektur hauptsächlich durch einbeziehen funktionaler parameter, nicht funktionale parameter wegen fehlender implementation schwer messbar.

\section{Ausblick}
Prozess in der Planungsphase erprobt. Implementierungsphase auch wichtig, aber nicht beschrieben. Nächste Schritte könnten sein den Prozess mit einer Implementierungsphase zu erweitern. 1 Projekt hat viele Probleme schon aufgezeigt, aber weitere, unterschiedliche Projekte wären gut um den Prozess noch zu verbessern, mehr Risiken zu betrachten, mehr analysen durchzuführen

Implementation sollte wahrscheinlich durch Prototyping gelöst werden um die Qualitätsattribute immer zu überprüfen




\clearpage
\bibliographystyle{gerabbrv}
\bibliography{Literatur}
\clearpage

% Das Abbildungsverzeichnis
\listoffigures
\clearpage

% Das Tabellenverzeichnis
\listoftables
\clearpage

\phantomsection
\addcontentsline{toc}{chapter}{Abkürzungsverzeichnis}
\chapter*{Abkürzungsverzeichnis}
\begin{acronym}[XXXXX]
    \acro{VPN}[VPN]{Virtual Private Network}
    \acro{IREB}[IREB]{Internation Requirements Engineering Board}
    \acro{CRUD}[CRUD]{Create Read Update Delete}
    \acro{ATAM}[ATAM]{Architecture Trade-off Analysis Method}
    \acro{CBAM}[CBAM]{Cost Benefit Analysis Method}
    \acro{ROI}[ROI]{Return of investment}
    \acro{UML}[UML]{Unified Modeling Language}
\end{acronym}
\clearpage

\phantomsection
\addcontentsline{toc}{chapter}{Anhang}
\chapter*{Anhang}


\end{document}
