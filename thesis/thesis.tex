\documentclass[Master,MSE,german]{twbook}
\usepackage[utf8]{inputenc}
\usepackage[T1]{fontenc}
\newcommand{\FHTWCitationType}{IEEE}
\usepackage{bibgerm}
\usepackage{float}

% Definition Code-Listings Formatierung:
\usepackage[final]{listings}
\lstset{captionpos=b, numberbychapter=false,caption=\lstname,frame=single, numbers=left, stepnumber=1, numbersep=2pt, xleftmargin=15pt, framexleftmargin=15pt, numberstyle=\tiny, tabsize=3, columns=fixed, basicstyle={\fontfamily{pcr}\selectfont\footnotesize}, keywordstyle=\bfseries, commentstyle={\color[gray]{0.33}\itshape}, stringstyle=\color[gray]{0.25}, breaklines, breakatwhitespace, breakautoindent}
\lstloadlanguages{[ANSI]C, C++, [gnu]make, gnuplot, Matlab}

\makeatletter
\renewcommand\lstlistingname{Quellcode}
\renewcommand\lstlistlistingname{Quellcodeverzeichnis}

% Definition des Macros listoflolentryname analog zu listoflofentryname und listoflotentryname der KOMA-Klasse
\newcommand\listoflolentryname\lstlistingname
% Neudefinition der Zeilen des Quellcodeverzeichnisses wenn die Option listof=entryprefix gewählt wurde
\@ifclasswith{scrbook}{listof=entryprefix}
{%
    \renewcommand\l@lstlisting[2]{\@dottedtocline{1}{1.5em}{1em}{\listoflolentryname~#1}{#2}}
}{%
}
\makeatother
\newcommand{\listofcode}{\phantomsection\lstlistoflistings}

%
% Einträge für Deckblatt, Kurzfassung, etc.
%
\title{Architekturerstellung auf Basis von Anforderungen}
\author{Bernhard Posselt, BSc}
\studentnumber{1310299032}
\supervisor{Mag. Maria-Therese Teichmann}
\place{Wien}
\kurzfassung{sehr kurz}
\schlagworte{Schlagwort1, Schlagwort2, Schlagwort3, Schlagwort4}
\outline{abstract da}
\keywords{Keyword1, Keyword2, Keyword3, Keyword4}

\begin{document}


\maketitle

%
% .. und hier beginnt die eigentliche Arbeit. Viel Erfolg beim Verfassen!
%

\chapter{Einführung}

\section{Motivation}
\subsection{Wie kommt man von Anforderungen auf eine gute Architektur}
Vermutung: Gute Architektur häng von guten Anforderungen ab, deswegen muss Anforderungsprozess erweitert werden
Probleme aufzählen die bei einer schlechten Architektur entstehen.

vielleicht noch genauer spezifizieren dass es keine all round solution werden soll sondern die meisten use cases abdecken soll (grafik wieviel apps SOA haben zb)

\subsection{Gibt es eine Art Kochrezept für die Architekturerstellung}
Gesucht: Ein Prozess mit dem auf die wichtigen Faktoren eingegangen werden kann.

\section{Was wird gemacht}
\subsection{Planung einer Zertifizierungsstellen Architektur}
\subsection{Modellierung mit UML}
\subsection{Anpassen der Anforderungestemplates}
\subsection{Architekturplanung Prozesserstellung, eine Art Framework}
\subsection{Dokumentation der Prozesserstellungsanläufe}

\section{Was wird nicht gemacht}
\subsection{Abdeckung des kompletten Prozesses}
Weil zu umfangreich, genauere Beschreibung im Architektur kapitel, auch erklären dass es sich heraus gestellt hat dass man nicht alles sofort planen und bewerten kann wegen fehlenden Messmöglichkeiten
\subsection{Kein Organisations- und Kommunikationsmanagement}
\subsection{Implementation}
Zu umfangreich
\subsection{Abdeckung aller möglichen Architekturfälle (Embedded, High Performance)}
\subsection{Abdeckung aller möglichen Archtekturreviewmethoden}
\subsection{Komplette Vorgaben der Bewertungs-Methoden}
Eher Hinweise auf wie man zb Ausfallkosten bewerten könnte, Methode auf eigene Anwendungsfälle abänder- und ersetzbar. Aber aufzeigen warum es gut sein kann sie dennoch schon zu behandeln
\subsection{Kompletter Anforderungsprozess}
Nur eine Erweiterung im Hinblick auf benötigte Parameter

\section{Übersicht}
Erklären was in welchen Kapiteln behandelt wird
\subsection{Modellierung}
\subsection{Anforderungen}
\subsection{Architektur}
\subsection{Prozesserstellung}
\subsection{Prozessumsetzung Anforderungen}
\subsection{Prozessumsetzung Architektur}
\subsection{Zusammenfassung}


\chapter{Modellierung}
Kurze Einführung in die Modellierung. Warum ist sie wichtig, was wird verwendet. Die wichtigsten Grundsachen erklären und erklären für was welcher Diagrammtyp verwendet wird

\section{Warum ist Modellierung wichtig}
\section{Was ist UML}

\section{Diagrammtypen}
\subsection{Kontextdiagramm}
\subsection{Komponentendiagram}
\subsection{Klassendiagramm}
\subsection{Aktivitätsdiagramm}
\subsection{Usecasediagramm}


\chapter{Anforderungen}
Behandeln der ISO 9126 Anforderungen, versuchen zu erklären was man alles beachten muss und was überhaupt ab anfang messbar ist (z.B. performance nicht messbar)

\section{Funktionale Anforderungen}
Wichtig: Security ist eine funktionale Anforderung

\section{Nicht funktionale Anforderungen}
\subsection{Reliability}
\subsection{Usability}
\subsection{Efficiency}
\subsection{Maintainability}
\subsection{Portability}

\chapter{Architektur}
Erklären generell was Software Architektur ist und mit was sie sich beschäftigt

\section{Architekturprozess Lifecycle}
Diagramm wie der Architekturprozess im groben abläuft. Dann den Teil markieren auf dem der Fokus der Arbeit liegt

\section{Was ist eine gute Architektur}
Zitate; Architektur ist ein abwägen von Vor- und Nachteilen. Es gibt keine perfekte Architektur, jede Entscheidung ist mit Vor und Nachteilen versehen.

\section{Architekturbewertungsmethoden}
Kurze Einführung in Architekturbewertungsmethoden

\subsection{ATAM}
\subsection{CBAM}

\chapter{Prozesserstellung}

sagen dass man eine art kochrezept will

\section{Vorhandene Daten}
Ausgangsparameter test
\subsection{Wer (Akteure}
\subsection{Was (Daten)}
\subsection{Warum (Anforderungen/Usecases)}
\subsection{Womit (Komponenten)}

\section{Prozesserstellungsversuche}
Ausgehend von den vorhandenen Daten wurden mehrere Prozesse definiert, welche bis auf den Letzten entweder zu grobe Ergebnisse lieferten, oder nicht nachvollziehbar waren.

Ausgangsbasis war eine Systemvision mit folgenden Anforderungen:

\begin{itemize}
  \item Es soll eine Webseite entstehen, welche die Prüfungstermine auflistet und Personen erlaubt, sich für diese Prüfungen anzumelden
  \item Die Übermittlung der Prüfungsdaten soll über einen eigenen VPN Server geschehen
  \item Die Prüfungsdaten werden firmenintern verwaltet und nach der Auswertung soll der Scheme Owner benachrichtigt werden
\end{itemize}

\begin{figure}[!htbp]
    \centering
    \includegraphics[scale=0.6]{uml/vision.png}
    \caption{Systemvision der Komponenten}
\end{figure}

Aufbauend darauf wurde dann versucht einen Prozess zu finden, der diese Grundideen berücksichtigt.

\subsection{Vom Usecase zur Komponente}
Der erste Versuch zur Erstellung des Architekturprozesses orientierte sich am Prinzip: teile und herrsche. Der Prozess hing stark von den nicht funktionalen Anforderungen ab und sollte auf folgender Weise funktionieren:

\begin{itemize}
  \item Für jeden Usecase wird ein komplettes Komponentendiagramm des Systems erstellt
  \item Die Komponenten jedes Teilsystems werden anhand ihrer nicht funktionalen Qualitäten aus einem Pool von Komponentenarchitekturen gewählt
  \item Schlussendlich werden alle Teilsysteme mit einander vereinigt, soweit es die nicht funktionalen Attribute erlauben
\end{itemize}

Dieser Prozess scheiterte nicht nur am enormen Modellierungsaufwand, sondern auch am Auswahlprozess der Komponenten: Je nachdem, welche Komponentenarchitekturen vorhanden waren und wie diese bewertet wurden, entstanden unterschiedliche Architekturen. Zudem schien es zu viele Komponentenarchitekturen zu geben, da die einzelnen Komponenten beliebig miteinander kombinierbar waren.

Die Qualität der Architektur hätte folglich von der Vollständigkeit dieser scheinbar unendlich großen Menge an Komponentenarchitekturen abgehangen. Aus diesem Grund schien er ungeeignet und wurde verworfen.

\subsection{Von einer geschätzten Architektur auf eine verfeinerte Architektur durch nicht funktionale Anforderungen}
Zweiter Versuch: Nach Erfahrungswerten so viel wie möglich erstellen, danach überprüfen auf nicht funktionale Anforderungen und notwenige Änderungen einbringen -> kein Kochrezept, verlässt sich zu viel auf Erfahrungswerte
\subsection{Von den Daten zur Architektur}
Dritter Versuch: Aufspalten der Architektur in Datenbereiche. Ging schon gut aber Ergebnis zu grob
\subsection{Von den Daten und den Akteuren zur Architektur}
Vierter: Einbeziehen der Akteure

\chapter{Prozess Anforderungen}
Da die Software Architektur auf den Anforderungen basiert und viel wichtiger \glqq den gestalterischen Spielraum des Architekten\grqq \cite[S. 103]{softarch} begrenzt, kann davon abgeleitet werden, dass sowohl die Qualität der Architektur als auch die Akzeptanz des Systems wesentlich von den bereits im Vorfeld ermittelten Parametern abhängt. Das bedeutet wiederum, dass für die Klärung der Ausgangsfrage - Wie kommt man von Anforderungen auf eine gute Architektur - auch der Anforderungsprozess eine wichtige Rolle spielt.

Da der Anforderungsprozess ein an sich eigenes, sehr großes Themengebiet dar stellt, wird hier jedoch nur auf die Ausgangsartefakte eingegangen, welche später im Architekturprozess referenziert werden.

Die Entwicklung der meisten Software Projekte ist oft ein dynamischer und agiler Prozess. Während des Entwicklungsprozesses werden oft Änderungen eingebracht und neue Anforderungen aufgestellt  \cite[S. 6-7]{effektiv}. Auch führt die Komplexität des Systems und das Wissen des/der KundIn dazu, dass nach der Erhebung der initialen Anforderungen nicht alle Parameter vollständig bekannt sind. Dies wiederum führt mehr oder weniger dazu, dass der Anforderungsprozess während der Entwicklung der Architektur nicht als abgeschlossen gesehen werden kann und der/die KundIn verfügbar sein muss. Dieses scheinbare Problem erlaubt es jedoch, den Aufwand der Aufnahme bestimmter Anforderungen zu minimieren, da bereits viele Kombinationen ausgeschlossen werden können.

\section{Ermittlung der Usecases}
Die Usecases werden zusammen mit dem/der Kundin ermittelt. Daraus wird schlussendlich ein Usecase Diagramm erstellt, welches alle Akteure und Nebensysteme beinhaltet. Dies ist wichtig für das Kontextdiagramm, welches auch im Anforderungsprozess erstellt wird und die Ausgangsbasis für die Architektur dar stellt.

\begin{figure}[!htbp]
    \centering
    \includegraphics[scale=0.4]{uml/usecase.png}
    \caption{Das mit dem Kunden ermittelte Usecase Diagramm}
\end{figure}

\subsection{Erweiterte Dokumenation der Usecases}
Parallel zur Erstellung des Usecase Diagramms werden zusätzliche Parameter und Beschreibungen für jeden Usecase aufgenommen, welche im eigentlichen Diagramm keine Erwähnung finden.
Dafür wird ein Anforderungstemplate, auch Usecase Beschreibung genannt, verwendet \cite[S. 214]{reqman}, welches aufbauend auf einer Grundversion \cite[Abbildung 8.14, S. 215]{reqman} für jeden Usecase folgende Angaben aufnimmt:

\begin{itemize}
  \item Id: eine eindeutige Bezeichnung, welche dafür verwendet wird, um den Usecase zu referenzieren
  \item Actor: eine Auflistung aller Teilnehmer des Usecases
  \item Description: eine kurze Beschreibung des Usecases
  \item Preconditions: eine Auflistung von Vorbedinungen für den Usecase
  \item Postconditions: eine Auflistung von Nachbedingungen für den Usecase
  \item Normal Course of Events: eine Beschreibung des Standardablaufes
  \item Alternative Courses: Auflistung von Erweiterungen oder zusätzlichen Möglichkeiten
  \item Exceptions: Beschreibung von diversen Ausnahmefälle
  \item Assumptions: Annahmen, unter welcher der Usecase beschrieben wird
  \item Priority: eine Gewichtung, wie wichtig der Usecase ist
  \item Notes: sonstige Anmerkungen
\end{itemize}

Sind die Abläufe komplexer, können Aktivitäts Diagramme verwendet werden, um komplexere Abläufe verständlicher dar zu stellen \cite[S. 215]{reqman}:

\begin{figure}[H]
    \centering
    \includegraphics[scale=0.4]{uml/takeexamreq.png}
    \caption{Der Ablauf des Take Exam Usecases im Detail}
\end{figure}

\subsection{Einbeziehen von Architekturreviewparametern}
Um die Einhaltung der Qualitätsparameter zu garantieren, können Architekturreviews durchgeführt werden \cite[S. 20]{review}. Es existiert zwar \glqq keine singuläre, allgemein akzeptierte Metrik um eine Architektur zu beurteilen\grqq \cite[S. 19]{review}, jedoch liefern sie grobe Einschätzungen über die Angemessenheit des Systems \cite[S. 20]{review}. Folgende Architekturreviews wurden dafür ausgewählt:

\begin{itemize}
  \item ATAM: betrachtet Wachstums- und explorative Szenarien um die Architektur zu beurteilen \cite[S. 61]{review}
  \item CBAM: basiert auf ATAM, beachtet jedoch vor allem den Nutzen und die Risiken und Kosten der Architketur, um die Architekturentscheidungen besser abwähgen zu können. Hauptfaktor ist der ROI. \cite[S. 67]{review}
\end{itemize}

Für diese Reviews können bereits früh ein Großteil der benötigten Parameter ermittelt bzw. zumindest grob abgeschätzt werden. Dies ist wichtig, weil nach der 10er Regel der Fehlerkosten früh erkannte Fehler und Probleme weniger Kosten nach sich ziehen als später Erkannte \cite[S. 154]{fehler}.

Deswegen wird das Anforderungstemplate um folgende Parameter erweitert:

\begin{itemize}
  \item Earned Value per month: Wieviel Umsatz der Usecase in einer bestimmten Zeit generiert
  \item Expected Usage: Anzahl der erwarteten Nutzer des Systems pro Zeiteinheit
  \item Growth Scenarios: Anzahl der erwarteten Nutzer des Systems pro Zeiteinheit bei einer höheren Nutzeranzahl
  \item Change Scenarios: mögliche Änderungsszenarien und Erweiterungen
\end{itemize}

\subsection{Einbeziehen von nicht funktionalen Qualitätsattributen}
Nicht funktionale Qualitätsattribute beschreiben die nicht funktionalen Anforderungen an das System. Da Qualität oft schwammig formuliert ist, ist es wichtig, diese Attribute messbar zu machen \cite[S. 9]{effektiv}.

Deswegen werden für jeden Usecase und dessen architekturrelevanten, nicht funktionalen Anforderungen messbare Parameter definiert. Diese Parameter helfen schon im Vorfeld dabei die Architektur zu überprüfen.

Für das Beispielprojekt wurden folgende Parameter definiert, mit welchen das Anforderungstemplate erweitert wurde:

\begin{itemize}
  \item Response Time in Seconds: Wie schnell die Antwort des Systems auf eine Anfrage reagieren muss
\end{itemize}

\section{Rahmenbedingungen}
Zusätzlich zu funktionalen und nicht funktionalen Anforderungen werden auch die Rahmenbedingungen ermittelt, unter welchem das System erstellt werden soll. Diese Anforderungen beinhalten meist den organisatorischen und zeitlichen Ablauf des Projektes und können auch gewisse Technologien vorschreiben, zB. wenn das System in ein bereits bestehendes System integriert werden soll. \cite[S. 9]{review}\cite[S. 110]{softarch}

Die Rahmenbedingungen des Beispielprojektes lassen sich zum Großteil aus dem ISO Standard für Zertifizierungsstellen ermitteln \cite{ISO_CERT} und geben Einsicht in die Vertraulichkeit der Daten und eröffnen weitere Usecases. Sofern möglich werden diese Parameter in das Usecase Diagramm und das Klassen Diagramm der zu verwendeten Daten mit einbezogen. Auf zeitliche und technologische Rahmenbedingungen wurde im Beispielprojekt nicht eingegangen.

\section{Ermittlung der Daten}
Die verwendeten Daten werden ermittelt und mit Hilfe eines Klassen Diagrammes modelliert. Dies ist nicht nur wichtig und nützlich, weil die Beispielanwendung in diesem Falle eine stark datenzentrierte Anwendung ist \cite[S. 105]{effektiv}, sondern wird später auch einen wesentlichen Beitrag zur Aufteilung des Systems in Komponenten leisten. Im Falle des Beispielprojekts wurden folgende Daten ermittelt und modelliert:

\begin{figure}[H]
    \centering
    \includegraphics[scale=0.5]{uml/class.png}
    \caption{Das ermittelte Klassendiagramm des Beispielprojektes}
\end{figure}

\section{Ermittlung der Zonen}
Nach der Ermittlung der Daten wird auf Basis der Rahmenbedingungen und Sicherheitsstruktur zusammen mit dem/der Kunden/Kundin ermittelt, welche Zonen für das System benötigt werden. Zonen stellen eigene abgeschlossene Bereiche dar, in welchen das System operiert oder von welchem auf das System zugegriffen werden kann. Grundsätzlich existiert mindestens eine Zone. In diesem Fall beherbergt diese Zone das interne System des Unternehmens.

Soll von anderen Systemen, zB. vom Internet auf Funktionalitäten des Systems zugegriffen werden können, wird eine weitere Zone benötigt.

Aus den Usecases des Beispielprojektes lässt sich ermittelt das in diesem Falle mindestens zwei Zonen benötigt werden:

\begin{itemize}
  \item Public: Usecases die vom Internet auf das System zugreifen, zB. wenn sich ein Anwärter (Applicant) für eine Prüfung registriert
  \item Internal: Usecases, welche nicht vom Internet aus zugänglich sind
\end{itemize}

Die Zonen werden nun mit einer Vertrautheitsebene von 1 bis n versehen, um eine Sicherheitshirarchie der Zonen zu generieren. Es kann vorkommen, dass mehrere Zonen mit der selben Vertrautheitsebene versehen werden, diese dürfen sich jedoch nicht überlappen. Zum Beispiel könnte eine weitere Tocherniederlassung der Firma ein eigenes System erstellen, auf welches man auch vom Internet aus zugreifen kann. Das Internet würde dann mit der Vertrautheitsebene 1 versehen und beide Systeme mit der Ebene 2. Überlappen sich diese Systeme, muss die überlappende Zone in ein eigenes System ausgegliedert werden.

Die ermittelten Daten werden nun den Zonen, in welchen sie essentiell für den Betrieb des Systemes sind, zugeteilt. Auch die Akteure werden in Zonen aufgeteilt, in welchen sie operieren. Scheinen Daten oder Akteure in mehreren Zonen auf, werden sie der Zone mit der höchsten Sicherheitsebene zugeteilt. Die niedrigeren Bereiche können dann auf diese Bereiche anhand von festgelegten APIs zugreifen.

Im Falle des Beispielprojektes werden alle aufgenommen Daten in der Internal Zone verwaltet, da alle Daten als businesskritisch für das Internal System angesehen werden, welches wiederum die höchste Sicherheitsebene besitzt. Würde das Beispielprojekt zB. um ein Forum oder einen Blog erweitert, würden diese der Public Zone zugeteilt werden; AkteurInnen des Internal Systems können zwar auch auf diese Forum zugreifen oder Blogeinträge schreiben, jedoch sind diese Aktivitäten und Daten nicht kritisch für den normalen Betrieb des Internal Systems. Würde das Forum oder der Blog gehackt werden, resultiert dies zwar in einem Schaden für das Unternehmen, das Internal System könnte aber auch weiterhin normal operiert werden.

Nach der generellen Aufteilung der Daten wird eine Analyse der Rahmenbedingungen durch geführt, um zu erfahren, ob bestimmte Daten aufgrund von Richtlinien besonders geschützt, und somit in eine eigene Zone mit einer höheren Vertrautheitsebene ausgelagert werden müssen. Das Beispielprojekt verlangt die vertrauliche Verarbeitung von Prüfungsergebnissen und erwähnt auch explizit das Personal der Prüfungsstelle \cite[7.3]{ISO_CERT}. Weil die Internal Zone, in welchem das Personal operiert, die Zone mit der höchsten Vertrautheitsebene darstellt, muss eine weitere Zone mit einer höheren Vertrautheitsebene erstellt werden. Diese Zone wird in diesem Falle mit der Vertrautheitsebene 3 versehen und unter der Beschreibung Confidential geführt.

Um diese Zonen besser zu visualisieren zu können, wird das UML Metamodel mit Hilfe eines Profiles angepasst. Jede Zonen erhält einen gleich lautetenden Stereotypen \cite[S. 518]{glasklar}:

\begin{figure}[H]
    \centering
    \includegraphics[scale=0.5]{uml/datastereotypes.png}
    \caption{Das Metamodell wird mit einem Profil um drei Stereotypen erweitert, welche die Zonen der Applikation darstellen}
\end{figure}

Zusätzlich werden diese Zonen auch mit Vertrautheitsebenen verbunden. Diese Ebenen werden im Profil mit einer Notiz versehen, welche die Ebene angibt.

\begin{figure}[H]
    \centering
    \includegraphics[scale=0.5]{uml/datastereotypeslevel.png}
    \caption{Die Zonen werden mit Vertrautheitsebenen versehen}
\end{figure}

Sind die Stereotypen erstellt und mit Vertrautheitsebenen versehen, kann nun damit begonnen werden, die AkteurInnen des Usecase Diagrammes und die Daten des Klassen Diagrammes mit diesen Stereotypen zu versehen.

\begin{figure}[H]
    \centering
    \includegraphics[scale=0.5]{uml/classstereotyped.png}
    \caption{Das Klassendiagramm wird mit Stereotypen der Vertraulichkeit erweitert}
\end{figure}

\begin{figure}[H]
    \centering
    \includegraphics[scale=0.4]{uml/stereotypedusecase.png}
    \caption{Die AkteurInnen des Usecase Diagrammes wird mit Stereotypen der Vertraulichkeit erweitert}
\end{figure}

\section{Ermittlung der Beziehungen zwischen Akteuren/Partnersystemen und Daten}
Auf Basis des Usecase Diagrammes können die Akteure und deren Partnersysteme mit Hilfe eines Kontext Diagrammes visualisiert werden. Im Gegensatz zum Usecase Diagramm geht das Kontext Diagramm auf die zwischen den Systemen und AkteurInnen fließenden Daten ein und stellt so die Verbindung der Daten und Nutzer auf.

\begin{figure}[H]
    \centering
    \includegraphics[scale=0.5]{uml/context.png}
    \caption{Das Kontext Diagramm zeigt das System, die Akteure und die Nachbarsysteme}
\end{figure}
\chapter{Prozess Architekturplanung}
Aufbauend auf den im Anforderungsprozess ermittelten Attribute, kann nun mit der Architekturplanung begonnen werden.

\section{Erstellen der Minimalen Architektur}
Das Kontextdiagramm, welches im Anforderungsprozess erstellt worden ist, zeigt das System mit allen AkteurInnen und Nachbarsystemen. Aufbauend darauf kann nun die minimale Architektur erstellt werden, welche sich aus dem System und den Nachbarsystemen ableitet.

\begin{figure}[H]
    \centering
    \includegraphics[scale=0.5]{uml/context.png}
    \caption{Das Kontextdiagramm liefert die Ausgangsbasis für die Architektur}
\end{figure}

Zuerst werden alle Datenflussnotizen entfernt. Danach werden alle Komponenten entfernt, welche kein eigenes System darstellen. In diesem Falle werden folgende Komponenten entfernt:

\begin{itemize}
  \item Applicant
  \item Certification Body
  \item Invigilator
\end{itemize}

Dies führt zu folgender Minimalarchitektur:

\begin{figure}[H]
    \centering
    \includegraphics[scale=0.7]{uml/minimalarch.png}
    \caption{Minimale Architektur}
\end{figure}

Für die Nachbarsysteme selbst wird keine Architektur erstellt, jedoch beeinflussen sie die Schnittstellen des Systems und sind deswegen wichtig für den weiteren Prozess. Sie werden in die Architektur einbezogen.

\section{Erstellen der Datenminimalarchitektur}
Auf Basis der im Anforderungsprozess ermittelten Zonen wird das System der vorher erstellte Minimalarchitektur in ebenso viele Teilsysteme unterteilt. Die Aktivitätsdiagramme werden an die neue Architektur angepasst: Für jedes Untersystem wird in den Diagrammen eine eigene Swimlane erstellt. Die involvierten AkteurInnen sind wenn möglich als eigene Swimlane modelliert, spielen in dieser Phase aber noch keine wichtige Rolle zur Gliederung des Systems.

\begin{figure}[H]
    \centering
    \includegraphics[scale=0.5]{uml/takeexamactivity1.png}
    \caption{Die Antworten werden nach der Prüfung an den Certification Body übermittelt. Der Request wird dann durch zwei Gateways zum finalen System geleitet.}
\end{figure}

Wechselt der Kontrollfluss eine Swimlane eines Systems, heißt dies, dass eine Verbindung zwischen den beiden sonst abgeschotteten Systemen benötigt wird. Dieses Verbindung wird als eigene Komponente modelliert und wird als Gateway bezeichnet. Die Aufgabe dieses Gateways ist es, folgende Attribute der Anfrage zu überprüfen und die Anfrage gegebenenfalls zu verwerfen oder weiterzuleiten:

\begin{itemize}
  \item Von welchem System kommt die Anfrage?
  \item Welches System ist das Ziel der Anfrage?
  \item Welche Schnittstelle dieses Systems ist das Ziel der Anfrage?
  \item Gibt es eine Regel die diese Anfrage explizit erlaubt?
\end{itemize}

Diese Komponenten fungieren damit als eine Art Application Firewall.

Die anfangs beschriebenen Nachbarsysteme werden nach ihren Anforderungen, welche aus den Aktivitätsdiagrammen ablesbar sind, entweder an das System in ihrer Zone angeschlossen, oder direkt mit dem Gateway verbunden.

Das Beispielprojekt bezieht Zahlungsdaten direkt von einem Payment System und die Prüfungsfragen werden direkt an das Scheme Owner System gesandt. Für diese beiden Usecases ist somit kein Zwischensystem notwendig und die Anfragen können vom Internal System direkt an das Ziel System geschickt werden. Dem entgegen gesetzt ist die Übermittlung der Prüfungsfragen des Scheme Owner Systems: Das System braucht eine durch Internet erreichbare Schnittstelle um die Daten zu übermitteln, da das Ausgangssystem nicht bekannt oder variabel ist und muss deswegen zuerst über das Public System zum Ziel.

\begin{figure}[H]
    \centering
    \includegraphics[scale=0.7]{uml/dataarch.png}
    \caption{Aufteilung der Komponenten in Datenbereiche}
\end{figure}

Wichtig ist hier, dass keine Gateways unterschiedlicher Vertrautheitsebenen übersprungen werden. Zeigt ein Aktivitätsdiagramm zB. einen Zugriff von Ebene 1 auf Ebene 3 muss dieser Zugriff sowohl durch den Gateway der Ebene 2 geleitet werden, als auch durch den Gateway der Ebene 3. Dies verhindert, dass besonders schützenswerte Systeme direkt an Systeme mit einer weitaus niedrigeren Vertrautheitsebene angeschlossen werden und so dessen Gateway zum Single Point of Failure wird. Dies gilt in beide Richtungen.

Da bei der Erstellung des Systems alle Schnittstellen und Systeme bekannt sind, können diese Regeln fest im Gateway verankert werden. Weil diese Gateways unabhängig voneinander agieren, können sie durch das Hinzufügen eines Load Balancers beliebig vervielfacht werden, was sowohl die Ausfallsicherheit als auch die Skalierbarkeit erhöht. Das ist wichtig, weil sie als einzige Verbindung zwischen den Systemen zu einer Art Flaschenhals werden.

\section{Einbinden der AkteurInnen}
Nachdem die Datenminimalarchitektur steht, können nun die AkteurInnen des Systems in die Aufgliederung des Systems mit einbezogen werden. Hierfür müssen nun die Objektflüsse und die AkteurInnen des Systems für jeden Usecase betrachtet werden, welche aus den vorher bereits erstellten Aktivitäts und Kontextdiagramm ersichtlich sind.

Zuerst wird das erste Untersystem, in diesem Falle das Public System, betrachtet. Alle Objektflüsse durch das System und die AkteurInnen, welche mit ihren Swimlanes angrenzen, sind in die Aktivitäten des Systems involviert. Jede Involvierung eines/einer Akteurs/Akteurin in ein System erfordert einen Zugang zu diesem System.

Jeder dieser Akteure muss mit den minimal möglichen Rechten für dieses System ausgestattet werden, um seine Aufgaben zu erfüllen. Dies vermeidet nicht nur Fehler sondern reduziert auch den Schaden, welcher ein potentieller Angriff dieses Akteurs/dieser Akteurin anrichten kann \cite[1. A]{leastpriv}.

Da ein System komplex ist \cite[S. 7]{softarch}, und diese Sicherheitsattribute nach Änderungen am System immer wieder überprüft werden müssen, stellt jeder zusätzliche Zugriff eines/einer Akteurs/Akteurin nicht nur ein Sicherheitsrisiko dar, sondern erhöht auch den Test- und damit den Wartungsaufwand. Idealerweise wird daher jedem/jeder AkteurIn ein eigenes, für sich abgekapseltes System zur Verfügung gestellt, was jedoch meist aufgrund Kosten der zusätzlichen Systeme keine Option dar stellt.

Um zu ermitteln, welche Systeme eine eigene Komponente benötigen, wird nun entweder anhand einer Tabelle oder zusammen mit dem/der KundIn pro Usecase und deren Komponenten ermittelt, ob der Schaden eines unerlaubten Zugriffs der Daten den eines Systems überschreitet. Die Schadens- und Systemkosten müssen zuerst von dem/der KundIn und dem/der ArchitektIn geschätzt werden.

Im Falle des Beispielprojektes wurde auf Basis des folgenden stark vereinfachten Aktivitätsdiagramms in Abbildung \ref{fig:actorarch} ermittelt, dass die möglichen Schadenskosten im Falle, dass der Anwärter (Applicant) Zugriff auf die Prüfungsantworten (Answer) bekommt, die eines eigenen Systems überschreiten. Das gleiche Problem trifft auch auf den Scheme Owner zu: die Schadenskosten im Falle einer Manipulation oder eines lesenden Zugriffes des Anwärters (Applicant) auf die Fragen überschreitet auch hier die Kosten eines eigenen Systems. Deswegen werden zwei zusätzliche Systeme erstellt und aus dem Public System ausgegliedert.

Für den Fall, dass bei der Aufspaltung zu viele Systeme entstanden sind, werden nun in einem weiteren Schritt diverse Kombinationen von Teilsystemen betrachtet und versucht zusammen zu legen, solange deren Schadenskosten nicht die Systemkosten überschreiten.

Beim Beispielprojekt zeigt sich, dass es keine erlaubte Kombination gibt, da die anderen AkteurInnen nicht auf diese Daten zugreifen dürfen. Es bleibt somit bei den ermittelten zwei Zusatzsystemen.

\begin{figure}[H]
    \centering
    \includegraphics[scale=0.6]{uml/actorarch.png}
    \caption{Vereinfachte Gegenüberstellung von Aktivitätsdiagramme für das Public System}
    \label{fig:actorarch}
\end{figure}

Diese Analyse wird für alle verbleibenden Systeme durchgeführt, bis alle Systeme aufgespalten sind.

Im Falle des Beispielprojektes führt dies schlussendlich zu folgender Systemaufspaltung:

\begin{figure}[H]
    \centering
    \includegraphics[scale=0.6]{uml/vision4.png}
    \caption{Architektur nach der Aufspaltung }
\end{figure}


\section{Modellieren der Komponenten Interfaces (Klassen Diagramm)}
Aufzeigen dass zb das interne System user anlegen können muss mit methoden im klassendiagramm

\section{Analyse der nicht funktionalen Attribute}
Auf Basis von dokumentierten Szenarien können nun nicht funktionale Attribute gemessen werden und Hinweise kritische/wichtige Komponenten gegeben werden. Kostengegenüberstellung können auch eigene Systeme rechtfertigen/entfernen

\subsection{Reliability}
Single Point of Failure Analyse (Matrix Komponente x Usecase), Erklären wie man auf Matrix kommt (Aktivitätsdiagramm), Ausfallskosten (inkl. Wachstumsszenarien)

Einfache Methode zur schätzung der Ausfallkosten, aufzeigen wie durch Reduzieren der Ausfallswahrscheinlichkeit Kosten sinken aber auch Investitionskosten verursachen. Wachstumsszenarien auch einbeziehen in die Rechnung
\subsection{Usability}
In diesem Teil des Prozesses nicht wichtig, da noch keine Implementierung vorhanden.

\subsection{Efficiency}
Efficiency kann pro usecase gemessen werden, zb für antwortzeiten indem man zb die swimlanewechsel der Aktivitätsdiagramme zählt und mit einer konstanten multipliziert (geschätzte Netzwerkgeschwindigkeit). Ansonsten ist es durch die fehlende Implementation nicht möglich die Geschwindigkeit oder den Arbeitsspeicherverbrauch zu messen.

\subsection{Maintainability}
Auslesbar aus der Usecase Matrix, als Summe aller subsysteme,

\subsection{Portability}
In diesem Teil des Prozesses nicht wichtig, da noch keine Implementierung vorhanden.


\section{Changemanagement}
Je nachdem ob noch möglich/gewünscht aufzeigen wie würde ein Change request ausschauen, z.B. was wäre nötig um einen zusätlichen Typ User des Systems einzubinden


\chapter{Zusammenfassung}

\section{Warum ist der Prozess gut}
\subsection{10er Regel der Fehlerkosten}
\subsection{Minimiert Angriffsfläche für wichtige Infrastruktur}
\subsection{Fokus Datensicherheit}
\subsection{Lässt Entscheidungen aufschieben}
\subsection{Kochrezept lässt sich ableiten}
\subsection{Wichtige Anforderungsparameter schon früh einbezogen}
\subsection{Visuelle Repräsentationen mit denen schnell bewertet werden kann}

\section{Limitierungen}
\subsection{Benötigt noch mehr Input von anderen Projekten}
da beispielprojekt wegen den anforderungen starken wert auf datensicherheit legt und deswegen architektur prozess beinflusst hat

\subsection{Nicht alle nicht funktionalen Anforderungen überprüfbar}
Sprich es ist bis zu einem gewissen Teil möglich
\subsection{Problem wenn sicht Akteure/Daten oft ändern}
\subsection{Zum Teil auch abhängig von Erfahrungswerten}
\subsection{Nur Plangungsphase, ohne Implementierungsphase}
\subsection{Extremarchitekturen}

\section{Erkenntnisse}
\subsection{Ohne messbare Parameter kein Kochrezept möglich}
\subsection{Parameter nicht zu jeder Phase messbar}
\subsection{Priorisierung von nicht funktionalen Parametern schwer möglich}
\subsection{Generische Komponentenarchitektur durch viele und schnell ändernde Kombinationen schwer möglich}
\subsection{Auf Ehrfahrungswerte kann nicht vollkommen verzichtet werden}
\subsection{Funktionale Anforderungen beeinflussen Archtitektur mehr als gedacht}
Prozess erstellt die frühe Architektur hauptsächlich durch einbeziehen funktionaler parameter, nicht funktionale parameter wegen fehlender implementation schwer messbar.

\section{Ausblick}
Prozess in der Planungsphase erprobt. Implementierungsphase auch wichtig, aber nicht beschrieben. Nächste Schritte könnten sein den Prozess mit einer Implementierungsphase zu erweitern. 1 Projekt hat viele Probleme schon aufgezeigt, aber weitere, unterschiedliche Projekte wären gut um den Prozess noch zu verbessern.

\clearpage
\bibliographystyle{gerabbrv}
\bibliography{Literatur}
\clearpage

% Das Abbildungsverzeichnis
\listoffigures
\clearpage

% Das Tabellenverzeichnis
\listoftables
\clearpage

% Das Quellcodeverzeichnis
\listofcode
\clearpage

\phantomsection
\addcontentsline{toc}{chapter}{Abkürzungsverzeichnis}
\chapter*{Abkürzungsverzeichnis}
\begin{acronym}[XXXXX]
    \acro{VPN}[VPN]{Virtual Private Network}
    \acro{CRUD}[CRUD]{Create Read Update Delete}
    \acro{ATAM}[ATAM]{Architecture Trade-off Analysis Method}
    \acro{CBAM}[CBAM]{Cost Benefit Analysis Method}
    \acro{ROI}[ROI]{Return of investment}
    \acro{WWW}[WWW]{world wide web}
\end{acronym}
\clearpage

\phantomsection
\addcontentsline{toc}{chapter}{Anhang}
\chapter*{Anhang}


\end{document}
